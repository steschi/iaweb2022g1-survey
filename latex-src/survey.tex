%----------------------------------------------------------------
%
%  File    :  survey.tex
%
%  Author  :  Keith Andrews, ISDS, TU Graz, Austria
%
%  Created :  24 Mar 2010
%
%  Changed :  22 Jan 2021
%
%----------------------------------------------------------------


\documentclass[11pt,onecolumn,twoside]{report}

\usepackage[
  a4paper,
  twoside,
  top=5mm,                % top margin
  bottom=7mm,             % bottom margin
  inner=20mm,             % inner margin (next to binding)
  outer=20mm,             % outer margin (opposite binding)
  bindingoffset=10mm,     % on binding side
  includeheadfoot,        % include head(er) and foot(er)
  headheight=10mm,        % height of header
  headsep=15mm,           % sep between header and text body
  footskip=15mm,          % sep between body and baseline of footer
  footnotesep = 10mm plus 2mm minus 0mm  % bottom of body to top of footnote
]{geometry}
% A4 paper is w=210m, h=297mm


\newcommand{\thistitle}{Responsive Data Visualization}  % title
\newcommand{\thissubject}{}  % leave empty for no subject
\newcommand{\thisauthor}{Thomas Pinheiro de Souza, Stefan Schintler and Andreas Steinkellner}          % author
% \newcommand{\thisauthor}{Keith Andrews, Tom Black, and Harry White}      % multiple authors
\newcommand{\thiskeywords}{survey paper, skeleton, guidelines, template}   % keywords

\newcommand{\thisdate}{04 Dec 2022}  % date of this version
\newcommand{\thisyear}{2022}         % year of this version




\newcommand{\fullh}{24cm}         % height of figures for 1 per page
\newcommand{\halfh}{9.5cm}        % height of figures for 2 per page
\newcommand{\thirdh}{6cm}         % height of figures for 3 per page


\setlength{\parindent}{1em}       % less indentation
\setlength{\parskip}{5pt plus 1pt minus 1pt}  % space before a paragraph


% \tolerance is set by LaTeX to 200
% \sloppy sets \tolerance = 9999
% which allows LaTeX more tolerance in adding word spacing

% \sloppy
% \fussy
% \tolerance = 1000

\tolerance=400 
% makes some lines with lots of white space, but      
% tends to prevent words from sticking out in the margin



\setcounter{tocdepth}{3}        % lowest section level entered in ToC
\setcounter{secnumdepth}{3}     % lowest section level still numbered




\usepackage[T1]{fontenc}        % 8-bit output chars (must be before inputenx)
\usepackage[utf8]{inputenx}     % input char encoding

\usepackage[english,austrian,british]{babel}

\usepackage{newtxtext}          % newer times fonts
\usepackage{newtxmath}

\usepackage{relsize}            % relative font sizes \smaller \larger
\usepackage{float}              % H for float placement
\usepackage{setspace}           % line spacing

\usepackage{textcomp}           % symbols such as \texttimes and \texteuro
\usepackage{latexsym}
\usepackage{fontawesome}        % fontawesome symbols
\usepackage{minted}

\usepackage{siunitx}            % prettier number formatting
\sisetup{%
  group-separator={,},
}
\usepackage[super]{nth}         % 1st, 2nd, 3rd, etc.

\usepackage{xspace}
\usepackage{xstring}            % string manipulation macros
\usepackage{xparse}             % commands with optional arguments
\usepackage{etoolbox}           % for \newrobustcmd
\usepackage{makecmds}           % for \makecommand
\usepackage{calc}               % for math calculations

\usepackage[svgnames,table,xcdraw]{xcolor}
\definecolor{darkgreen}{rgb}{0.0,0.2,0.0}
\definecolor{darkblue}{rgb}{0.0,0.0,0.2}
\definecolor{darkred}{rgb}{0.2,0.0,0.0}
\definecolor{verylightgrey}{gray}{0.95}
\definecolor{lightgrey}{gray}{0.9}
\definecolor{grey}{gray}{0.7}
\definecolor{black}{gray}{0.0}

\definecolor{tableheadercolour}{gray}{0.8}
\definecolor{tablerowcolour}{gray}{0.93}

\usepackage{longtable}
\usepackage{multirow}
\usepackage{tabularx}

% Define some new column types for tables:
% like X but flushleft (= raggedright) rather than justified
\newcolumntype{Y}{>{\raggedright\arraybackslash}X}
% a p column but flushleft (= raggedright) rather than justified
\newcolumntype{L}[1]{>{\raggedright\arraybackslash}p{#1}}
% a p column but flushright (= raggedleft) rather than justified
\newcolumntype{R}[1]{>{\raggedleft\arraybackslash}p{#1}}


\usepackage{booktabs}           % nicer tables

\newcommand{\tablestretch}
{\renewcommand{\arraystretch}{1.20}}  % spacing between table rows




\usepackage{verbdef}            % define robust verb strings
\usepackage{verbatim}
\usepackage{comment}



% better lists
\usepackage{enumitem}

\setlist{
  topsep=0pt,
  partopsep=0pt,
  parsep=0.6ex,
  itemsep=1.2ex,
  left=\parindent .. 2\parindent,    % bullet .. start ot text
}

\setlist[description]{
  style=sameline,
}




\usepackage{listings}                 % for listings of source code

\makeatletter
\newlength{\numwidth}%
\setlength{\numwidth}{\widthof{\normalfont{\lst@numberstyle{99}}}}% Up to 2-digit (99) line numbers
\def\lst@PlaceNumber{%
  \makebox[\numwidth+1em][l]{%
    \makebox[\numwidth][r]{\normalfont\lst@numberstyle{\thelstnumber}}%
  }%
}
\makeatother

% lstset strategy: define defaults here for
% all non-floating (displayed) listings
% floated listings override these settings later

\lstset{                              % set parameters for listings
  floatplacement=tp,                  % default float placement
  numberbychapter,
  inputencoding=utf8,
  language=,                          % empty = plain text
  basicstyle=\small\ttfamily,
  tabsize=2,
  xleftmargin=2\parindent,
  xrightmargin=2\parindent,
  frame=none,
  framexleftmargin=0mm,
  rulesepcolor=\color{verylightgrey},
  numbers=none,
  numberstyle=\scriptsize,
  numbersep=2ex,
  breaklines,
  showtabs=false,
  showspaces=false,
  showstringspaces=false,
  keywordstyle=\color{black},
  commentstyle=\color{SteelBlue},
  identifierstyle=,
  stringstyle=,
  captionpos=b,
  abovecaptionskip=\abovecaptionskip,
  belowcaptionskip=\belowcaptionskip,
  extendedchars=true,           % listings usually only support 7-bit ascii chars
  literate=%                    % map some one-byte utf8 chars for use in listings
%    { }{{~}}1                   % non-breaking space
    {©}{{\textcopyright}}1
    {€}{{\texteuro}}1
    {Ö}{{\"O}}1
    {Ä}{{\"A}}1
    {Ü}{{\"U}}1
    {ß}{{\ss}}1
    {ö}{{\"o}}1
    {ä}{{\"a}}1
    {ü}{{\"u}}1,
}


\lstdefinelanguage{biblatex}   % based on biblatex v 2.7a from 2013-07-14
{
  keywords={%
    @article,@book,@mvbook,@inbook,@bookinbook,@suppbook,%
    @booklet,@collection,@mvcollection,@incollection,@suppcollection,%
    @manual,@misc,@online,@patent,@periodical,@suppperiodical,%
    @proceedings,@mvproceedings,@inproceedings,@reference,@mvreference,%
    @inreference,@report,@set,@thesis,@unpublished,@xdata,%
    @conference,@electronic,@mastersthesis,@phdthesis,@techreport,@www,%
    @artwork,@audio,@bibnote,@commentary,@image,@jurisdiction,@legislation,%
    @legal,@letter,@movie,@music,@performance,@review,@software,%
    @standard,@video%
  },
  sensitive=false,
  comment=[l][\itshape]{@comment},
  morecomment=[l]{\%},
}

\lstdefinelanguage{CSS}
{
  alsoletter={-},
  morekeywords={%
  color,background,background-color,margin,padding,font,
  font-family,weight,%
  display,position,top,left,right,bottom,list,%
  style,border,size,white,space,min,width%
  },
  sensitive=false,
  morecomment=[l]{//},
  morecomment=[s]{/*}{*/},
  morestring=[b]",
}





\usepackage[compact,nobottomtitles,pagestyles,explicit]{titlesec}
% when using explicit, must explicitly include #1 for titlename

% nobottomtitles
% move section headings close to page bottom to next page
\renewcommand{\bottomtitlespace}{2cm}

% \chaptermark sets the value of \chaptertitle for later
% \@chapapp is defined as \chaptername outside the appendix,
% and as \appendixname within the appendix.
\makeatletter
\titleformat{\chapter}
[display]                                            % shape
{\chaptermark{\thechapter~~#1}\sffamily\bfseries}    % format
{\huge\@chapapp\ \thechapter}                        % label
{4ex}                                                % sep
{\Huge#1}                                            % before-code
\makeatother

\titleformat{name=\chapter,numberless}
[block]                                              % shape
{\chaptermark{#1}\sffamily\bfseries}                 % format
{}                                                   % label
{0ex}                                                % sep
{\Huge#1}                                            % before-code

\titleformat{\section}
{\normalfont\Large\sffamily\bfseries}{\thesection}{0.8em}{#1}

\titleformat{\subsection}
{\normalfont\large\sffamily\bfseries}{\thesubsection}{0.8em}{#1}

\titleformat{\subsubsection}
{\normalfont\normalsize\sffamily\bfseries}{\thesubsubsection}{0.8em}{#1}

\titleformat{\paragraph}[runin]
{\normalfont\normalsize\sffamily\bfseries}{\theparagraph}{0.8em}{#1}

\titleformat{\subparagraph}[runin]
{\normalfont\normalsize\sffamily\bfseries}{\thesubparagraph}{0.8em}{#1}


% vertical spacing before and after section titles
\titlespacing*{\section}
{0pt}{3.5ex plus 0.5ex minus 0.5ex}{0ex plus 0ex minus 0.2ex}

\titlespacing*{\subsection}
{0pt}{2.5ex plus 0.5ex minus 0.5ex}{0ex plus 0ex minus 0.2ex}

\titlespacing*{\subsubsection}
{0pt}{2ex plus 0.5ex minus 0.5ex}{0ex plus 0ex minus 0.2ex}


% define page headings how I want them

\newpagestyle{main}[\small]{
% \addtolength\headheight{6.7pt}
% \headrule
\sethead%
[{\parbox[t]{0.3\textwidth}%                    % even left
  {\sffamily\thepage}}]
[]%                                             % even centre
[{\parbox[t]{0.6\textwidth}%                    % even right
  {\raggedleft\sffamily\chaptertitle}}]
{{\parbox[t]{0.6\textwidth}%                    % odd left
  {\sffamily\sectiontitle}}}%
{}%                                             % odd centre
{{\parbox[t]{0.3\textwidth}%                    % odd right
  {\raggedleft\sffamily\thepage}}}
}




\usepackage{titletoc}

% \contentsmargin{2.55em}

\titlecontents{chapter}%
[1.5em]%                         % left indent to entry text
{\addvspace{1em}\bfseries}%      % above-code per entry
{\contentslabel{1.5em}}%         % format for numbered entry
{\hspace*{-1.5em}}%              % format for unnumbered entry
{\hfill\contentspage}%           % [no dots] and page num per entry


% Note: \dottedcontents is short form of \titlecontents

\dottedcontents{section}%
[3.8em]%                         % left indent to entry text = 1.5 + 2.3
{}%                              % above-code per entry
{2.3em}%                         % label width
{1pc}%                           % space around the dots

\dottedcontents{subsection}%
[7.4em]%                         % left indent to entry text = 3.8 + 3.6
{}%                              % above-code per entry
{3.6em}%                         % label width
{1pc}%                           % space around the dots


\dottedcontents{figure}%         % LoF entries
[3.0em]%                         % left indent to entry text = 3.8 + 3.6
{}%                              % above-code per entry
{3.0em}%                         % label width
{1pc}%                           % space around the dots

\dottedcontents{table}%          % LoT entries
[3.0em]%                         % left indent to entry text = 3.8 + 3.6
{}%                              % above-code per entry
{3.0em}%                         % label width
{1pc}%                           % space around the dots



% List of Listings is unknown to titletoc, define here

% Add extra per-chapter space to LoL to mimic LoF and LoT
% (requires package etoolbox)
\makeatletter
\patchcmd{\@chapter}% <cmd>
  {\addtocontents}% <search>
  {\addtocontents{lol}{\protect\addvspace{10\p@}}% add per-chapter space
   \addtocontents}% <replace>
  {}{}% <success><failure>
\makeatother

% Configure LoL to mimic LoF and LoT
\contentsuse{lstlisting}{lol}

\titlecontents{lstlisting}%
[3.0em]%                              % left indent
{\addvspace{1.5mm}}%                  % above-code per entry
{\contentslabel{3.0em}}%              % format for numbered entry
{\hspace*{-3.0em}}%                   % format for unnumbered entry
{\titlerule*[1pc]{.} \contentspage}%  % dots and page num per entry
[]%                                   % below-code per entry

\renewcommand{\lstlistlistingname}{List of Listings}






% sensible settings for floats

\setlength{\textfloatsep}{9mm plus 2mm minus 2mm}
\setlength{\floatsep}{9mm plus 2mm minus 2mm}
\setlength{\intextsep}{9mm plus 2mm minus 2mm}

\setlength{\dbltextfloatsep}{9mm plus 2mm minus 2mm}
\setlength{\dblfloatsep}{9mm plus 2mm minus 2mm}

\setlength{\abovecaptionskip}{4mm plus 2mm minus 1mm}
\setlength{\belowcaptionskip}{2mm plus 1mm minus 1mm}


% See http://www-rohan.sdsu.edu/~aty/bibliog/latex/floats.html
% See https://robjhyndman.com/hyndsight/latex-floats/

\setcounter{topnumber}{2}               % max num floats at top of page
\setcounter{dbltopnumber}{2}            % max num floats on 2col page
\setcounter{bottomnumber}{2}            % max num floats at bottom of page
\setcounter{totalnumber}{4}             % max num floats on a page

\renewcommand{\topfraction}{0.8}        % max fraction of floats at top
\renewcommand{\dbltopfraction}{0.9}     % max fraction of floats at top 2col
\renewcommand{\bottomfraction}{0.8}     % max fraction of floats at bottom
\renewcommand{\textfraction}{0.2}       % min fraction of text

% only for entirely float pages:
\renewcommand{\floatpagefraction}{0.7}      % min page fraction having floats
\renewcommand{\dblfloatpagefraction}{0.7}   % min 2col page fraction having floats


% \usepackage[section,above,below]{placeins}  % keep floats to their own section




% use caption and subfig (caption2 and subfigure are now obsolete)

\usepackage[
  position=bottom,
  margin=1cm,
  font=small,
  labelfont={bf,sf},
  format=plain,
  indention=5mm,
  aboveskip=4mm,
  belowskip=0mm,
]{caption,subfig}

\captionsetup[subfigure]{
  margin=0pt,
  parskip=0pt,
  indention=5mm,
  farskip=4mm,            % skip above subfig (assuming captions at bottom)
  captionskip=2mm,        % skip between subfig and subcaption
}




\usepackage[short]{datetime}   % load datetime *after* babel, requires fmtcount
% \newdateformat{britdate}{%
% \ordinaldate{\THEDAY} \,\monthname[\THEMONTH] \THEYEAR
% }
\newdateformat{unixdate}{%
\twodigit{\THEDAY}~\shortmonthname[\THEMONTH]~\THEYEAR
}



\usepackage[
  autostyle=true,          % adapt quote style to current language
  english=british,         % british english as default
  threshold=1,             % set block quotations >1 line in display mode
  maxlevel=4,              % max nesting level
]{csquotes}

\usepackage[
  indentfirst=false,
  vskip=0pt,               % by default would be \topsep + \partopsep.
]{quoting}

% tell csquotes to use quoting environment
% for \displayquote and \blockquote
\SetBlockEnvironment{quoting}

% if cite is issued by a csquote command
\renewcommand{\mkcitation}[1]{\space#1}

% I prefer double quotes as outer
\DeclareQuoteStyle{keithbritish}%  [variant]{style}
  {\textquotedblleft}%                      opening outer mark
  {\textquotedblright}%                     closing outer mark
  [0.05em]%
  {\textquoteleft}%                         opening inner mark
  {\textquoteright}%                        closing inner mark

\ExecuteQuoteOptions{style=keithbritish}





\usepackage[
  backend=biber,
%  style=ext-authoryear-comp,   % defined in biblatex-ext package
  style=ext-authoryear,        % defined in biblatex-ext package
  sorting=nyt,
  useprefix,                   % van and von are part of second name
  mergedate=false,             % only for authoryear style
  dashed=false,                % only for authoryear style
  abbreviate=false,
  maxcitenames=2,              % if > 2 authors,
  mincitenames=1,              % use first 1 then et al
  maxbibnames=99,              % if > 99 authors,
  minbibnames=6,               % use first 6 then et al
  uniquelist=minyear,
  uniquename=init,
  hyperref=true,
  backref=true,
  backrefstyle=two,
  sortlocale=en,
]{biblatex}



% set for csquotes, but \autocite only available
% after biblatex is loaded
\SetCiteCommand{\autocite}    % or maybe \parencite

% more space between entries in bib
\setlength\bibitemsep{1.5\itemsep}

% kandrews: replace round brackets with square brackets in citations
\DeclareOuterCiteDelims{parencite}{\bibopenbracket}{\bibclosebracket}
\DeclareInnerCiteDelims{textcite}{\bibopenbracket}{\bibclosebracket}

% kandrews: replace round brackets with square brackets in bibliography
% biblabeldate is a biblatex-ext feature
\DeclareFieldFormat{biblabeldate}{\mkbibbrackets{#1}}


% remove URL: from in front of URLs
\DeclareFieldFormat{url}{\url{#1}}
\DeclareFieldFormat{doi}{\doi{#1}}
\DeclareFieldFormat{isbn}{\isbn{#1}}
\DeclareFieldFormat{issn}{\issn{#1}}

% suppress urldate field
\AtEveryBibitem{\clearfield{urlyear}}

% remove In: from @article and @inproceedings entries
% https://tex.stackexchange.com/questions/10682/suppress-in-biblatex
\renewbibmacro{in:}{%
  \ifboolexpr{%
     test {\ifentrytype{article}}%
     or
     test {\ifentrytype{inproceedings}}%
  }{}{\printtext{\bibstring{in}\intitlepunct}}%
}

% make all entry titles italic
% (also removes quotation marks from around titles)
% https://tex.stackexchange.com/questions/311816/want-title-in-simple-numeric-not-italic-through-bibliography
\DeclareFieldFormat*{title}{\mkbibitalic{#1}}
\DeclareFieldFormat*{citetitle}{\mkbibitalic{#1}}

% make journal names non-italic
\DeclareFieldFormat{journaltitle}{#1\isdot}

% make proceedings names non-italic
\DeclareFieldFormat[inproceedings]{booktitle}{#1\isdot}

% use nth for edition
\DeclareFieldFormat{edition}{%
  \ifinteger{#1}
    {\nth{#1}~\bibstring{edition}}
    {#1\isdot}}

% overwrite some standard strings in english.lbx
\DefineBibliographyStrings{english}{%
  edition          = {Edition},
  mathesis         = {Master's Thesis},
  phdthesis        = {PhD\addabbrvspace Thesis},
}


% kandrews
% use Unix format for dates in biblio:
% 29 Dec 2015, 01 Oct 2018, etc.

% for now, define under lang english not british
% due to bug in biblatex 3.11

\DefineBibliographyStrings{english}{%
  january          = {Jan},
  february         = {Feb},
  march            = {Mar},
  april            = {Apr},
  may              = {May},
  june             = {Jun},
  july             = {Jul},
  august           = {Aug},
  september        = {Sep},
  october          = {Oct},
  november         = {Nov},
  december         = {Dec},
}

\DefineBibliographyExtras{english}{%
% #1 = year, #2 = month, #3 = day
\protected\def\mkbibdatelong#1#2#3{%
  \iffieldundef{#3}
    {}
    {\mkdayzeros{\thefield{#3}}%
     \iffieldundef{#2}{}{\nobreakspace}}%
  \iffieldundef{#2}
    {}
    {\mkbibmonth{\thefield{#2}}%
     \iffieldundef{#1}{}{\space}}%
  \iffieldbibstring{#1}{\bibstring{\thefield{#1}}}{\mkyearzeros{\thefield{#1}}}}%
%
\protected\def\mkbibdateshort#1#2#3{%
  \iffieldundef{#3}
    {}
    {\mkdayzeros{\thefield{#3}}%
     \iffieldundef{#2}{}{\nobreakspace}}%
  \iffieldundef{#2}
    {}
    {\mkbibmonth{\thefield{#2}}%
     \iffieldundef{#1}{}{\space}}%
  \iffieldbibstring{#1}{\bibstring{\thefield{#1}}}{\mkyearzeros{\thefield{#1}}}}%
}



\addbibresource{ref.bib}



% xurl provides better URL breaking than url
% load after biblatex
\usepackage[hyphens,obeyspaces]{xurl}
\def\UrlFont{\smaller\ttfamily}






\usepackage{ifpdf}

\ifpdf
  % pdflatex
  \usepackage[pdftex]{graphicx}
  \DeclareGraphicsExtensions{.pdf,.jpg,.png}
  \pdfcompresslevel=9
  \pdfobjcompresslevel=1  % also compress PDF object streams except embedded PDFs
  \pdfpageheight=297mm
  \pdfpagewidth=210mm
  \usepackage[         % hyperref should be last package loaded
    unicode,
    pdftex,
    pdfversion=1.7,
    pdftitle={\thistitle},
    pdfsubject={\thissubject},
    pdfauthor={\thisauthor},
    pdfkeywords={\thiskeywords},
    bookmarks,
    bookmarksnumbered,
    linktocpage,
    colorlinks,
    linkcolor=darkred,
    anchorcolor=red,
    citecolor=darkgreen,
    urlcolor=darkblue,
    pdfstartview=Fit,              % initial view
    pdfview=Fit,                   % view after following a link
    pdfpagelayout=SinglePage,      % single page, no scrolling
    pdfpagemode=UseOutlines,       % open bookmarks in Acrobat
    plainpages=false,              % avoids duplicate page number problem
    pdfpagelabels,                 % avoids duplicate page number problem
    breaklinks=true,               % allow links exceeding a single line
  ]{hyperref}

\else
  % latex
  \usepackage[dvips]{graphicx}
  \DeclareGraphicsExtensions{.eps}
  \usepackage[dvips]{hyperref}
\fi


% export adjustbox keys to includegraphics
% must be after \usepackage{graphicx}
\usepackage[export]{adjustbox}    % valign=t, frame, ...





% subset of macros from thesis-macros

% \liintro list item intro is a style used when list items have an
% introduction phrase (say in italics) followed by a colon.
\newcommand{\liintro}[1]{\emph{#1}}

% short notes in square brackets
\newcommand{\shortnote}[1]
{%
{{\smaller{}[#1]}}
}


\newcommand{\TODO}[1]
{
{\textcolor{red}{[TODO: #1]}}
}



\newcommand{\imgcredit}[1]
{\smaller{}[#1]}



\newcommand{\copyrightACM}
{%
Copyright \copyright\ by the Association for Computing Machinery, Inc.%
}




\newcommand{\daymonthyear}[3]
{%
\twodigit{#1}\hspace{0.7ex}\nolinebreak[2]\shortmonthname[#2]\hspace{0.7ex}\nolinebreak[2]#3%
}


\newcommand{\monthyear}[2]
{%
\shortmonthname[#1]\hspace{0.7ex}\nolinebreak[2]#2%
}


\newcommand{\yearmonthday}[3]
{%
\twodigit{#3}\hspace{0.7ex}\nolinebreak[2]\shortmonthname[#2]\hspace{0.7ex}\nolinebreak[2]#1%
}


\newcommand{\yearmonth}[2]
{%
\shortmonthname[#2]\hspace{0.7ex}\nolinebreak[2]#1%
}



% link to Amazon or
% http://worldcatlibraries.org/wcpa/isbn/[ISBN number]
% http://amazon.com/exec/obidos/ASIN/#1/keithandrewshcic
% http://amazon.com/dp/#1/

\newrobustcmd{\isbn}[1]
{%
{%
\ifpdf
{\smaller ISBN
\href{http://amazon.co.uk/dp/#1/}{#1}}%
\else
{\smaller ISBN #1}%
\fi
}%
}



% ISSN
% http://www.bl.uk/services/bibliographic/issn.html
% 8 digits, should be printed xxxx-xxxx
% e.g. 0020-0190 is Information Processing Letters, Elsevier
%
% Lookup services:
% http://kmittlib.lib.kmutt.ac.th:81/search/i?SEARCH=0020-0190
% http://worldcatlibraries.org/wcpa/issn/0020-0190

\newrobustcmd{\issn}[1]
{%
{%
\ifpdf
{\smaller ISSN
\href{http://worldcatlibraries.org/wcpa/issn/#1}{#1}}%
\else
{\smaller ISSN #1}%
\fi
}%
}



% DOIs  http://doi.org/  e.g.
% doi:10.1038/nature723
% http://doi.org/10.1038/nature723

\newrobustcmd{\doi}[1]
{%
{%
\def\UrlFont{\smaller\rmfamily}
\ifpdf                                   % pdflatex
\href{http://doi.org/#1}{doi:\protect\nolinkurl{#1}}%
\else                                    % latex
doi:\protect\nolinkurl{#1}%
\fi
}%
}





\newrobustcmd{\website}[1]
{%
\ifpdf                                  % pdflatex
\href{http://#1/}{\nolinkurl{#1}}%
\else                                   % latex
\nolinkurl{#1}%
\fi
}




\newcommand{\news}[1]
{%
\ifpdf
\href{news:#1}{\nolinkurl{#1}}
\else
\nolinkurl{#1}%
\fi
}








% based on url package
% define styles for class, file, and variable names
% which break nicely at line breaks

% make the macros robust so they work inside captions, etc

\newcommand{\ttname}{\begingroup \smaller\urlstyle{tt}\Url}
\newcommand{\rmname}{\begingroup \smaller\urlstyle{rm}\Url}
\newcommand{\sfname}{\begingroup \smaller\urlstyle{sf}\Url}


% fname is for file names and directory names
\newrobustcmd{\fname}[1]{\ttname{#1}}

% vname is for variable names, domain names, email addresses
\newrobustcmd{\vname}[1]{\ttname{#1}}




% for class names, define our own url style

\makeatletter  % protect @ names

% \url@letstyle: New URL style to premit break at any letters.
% Based on \url@ttstyle

\def\Url@letdo{% style assignments for tt fonts or T1 encoding
\def\UrlBreaks{\do\a\do\b\do\c\do\d\do\e\do\f\do\g\do\h\do\i\do\j\do\k\do\l%
               \do\m\do\n\do\o\do\p\do\q\do\r\do\s\do\t\do\u\do\v\do\w\do\x%
               \do\y\do\z%
               \do\A\do\B\do\C\do\D\do\E\do\F\do\G\do\H\do\I\do\J\do\K\do\L%
               \do\M\do\N\do\O\do\P\do\Q\do\R\do\S\do\T\do\U\do\V\do\W\do\X%
               \do\Y\do\Z%
}%
\def\UrlBigBreaks{\do\.\do\@\do\\\do\/\do\!\do\_\do\|\do\%\do\;\do\>\do\]%
 \do\)\do\,\do\?\do\'\do\+\do\=\do\#\do\:\do@url@hyp}%
\def\UrlNoBreaks{\do\(\do\[\do\{\do\<}% (unnecessary)
\def\UrlSpecials{\do\ {\ }}%
\def\UrlOrds{\do\*\do\-\do\~}% any ordinary characters that aren't usually
\Urlmuskip = 0mu plus 1mu%
}

\def\url@letstyle{%
\@ifundefined{selectfont}{\def\UrlFont{\sf}}{\def\UrlFont{\sffamily}}\Url@letdo
}

\makeatother  % unprotect @ names

% class names
\newcommand\letname{\begingroup \smaller\urlstyle{let}\Url}

\newrobustcmd{\cname}[1]{\letname{#1}}


% ui element names
\newrobustcmd{\uiname}[1]{{\smaller\textsf{#1}}}

% html5 element names
\newrobustcmd{\elname}[1]{{\lstinline{<#1>}}}

% css class names
\newrobustcmd{\cssname}[1]{{\lstinline{#1}}}



% Euro symbol
\newcommand{\euro}{\texteuro\,}

% times symbol
\newcommand{\timessym}{\texttimes\,}

% approx symbol
\newcommand{\approxsym}{\ensuremath\approx\,}

% plusminus symbol
\newcommand{\plusminussym}{\textpm\,}

% not equal symbol
\newcommand{\neqsym}{\ensuremath{\neq\,}}

% rightarrow symbol
\newcommand{\rightarrowsym}{\ensuremath\rightarrow\,\,}


% thumbs up and thumbs down symbols

\newcommand{\uthumb}{\smaller[2]\raisebox{1pt}{\textcolor{DarkGreen}{\faThumbsUp}}}

\newcommand{\dthumb}{\smaller[2]\raisebox{1pt}{\textcolor{DarkRed}{\faThumbsDown}}}

% TODO REMOVE AFTER FILLING WITH PROPER STUFF %
\usepackage{lipsum}





\begin{document}

\unixdate

\normalsize
\pagestyle{empty}         % for preliminary pages (no numbers shown)
\pagenumbering{Roman}     % for pdf labels




\begin{titlepage}

\begin{center}

\begin{spacing}{1.1}
\Large\sffamily\bfseries
\thistitle
\end{spacing}

\ifstrempty{\thissubject}{}%     % if empty subject string, do nothing
{%
\begin{spacing}{1.1}
\large\sffamily\bfseries
\thissubject
\end{spacing}
}


\vspace{1cm}

{\large\sffamily \thisauthor}

% {\large\sffamily Group 4}
% \vspace{5mm}
% {\large\sffamily Keith Andrews, Tom Strong, Bill Weak, and Seb Green}

\vspace{1cm}

% Institute of Interactive Systems and Data Science (ISDS), \\
% Graz University of Technology \\
% A-8010 Graz, Austria \\[1cm]


{\large
706.041 Information Architecture and Web Usability WS 2022\\
Graz University of Technology
}


\vspace{1cm}

\thisdate

\end{center}



\vspace{2cm}

\begin{quote}
%\begin{center}
{\large\sffamily\bfseries Abstract}
TODO! This is where we'll put our abstract.
%\end{center}

\end{quote}

\vfill

\begin{center}
{\footnotesize\sffamily \copyright~Copyright \thisyear{} by the author(s),
except as otherwise noted.}

\vspace{2mm}
{\footnotesize\sffamily This work is placed under a
Creative Commons Attribution 4.0 International
(\href{https://creativecommons.org/licenses/by/4.0/}{CC BY 4.0}) licence.}
\end{center}

\end{titlepage}




\cleardoublepage
\pagestyle{plain}             % for preliminary pages
\pagenumbering{roman}         % for preliminary pages


\begin{spacing}{0.8}
\tableofcontents
\end{spacing}
\addcontentsline{toc}{chapter}{Contents}

\cleardoublepage
\begin{spacing}{0.8}
\listoffigures
\end{spacing}
\addcontentsline{toc}{chapter}{List of Figures}



\cleardoublepage
\pagestyle{main}              % for main pages
\pagenumbering{arabic}        % for main pages


\cleardoublepage
%----------------------------------------------------------------
%
%  File    :  intro.tex
%
%  Authors :  Pinheiro de Souza, Schintler, Steinkellner
% 
%  Created :  22 Nov 2022
% 
%----------------------------------------------------------------


\chapter{Introduction}

\label{chap:Intro}
Web graphics are a way to include visual scenes on websites or web applications.
While static images can be easily included to an existing site by using the HTML image tag,
interactivity is often a crucially important feature needed to bring ideas or scenes to life.
It can have a positive impact on overall site engagement \parencite{web-engagement-literature}.
Interactivity can be a very broad term to define, but it includes anything a user can do within
a graphic to manipulate the scene. \\

The complete code of all examples mentioned in this report can be examined on GitHub \parencite{github-repo}.

\section{Two-Dimensional Web Graphics}
The traditional two-dimensional approaches of including interactive
web graphics to a website use either SVG or Canvas2D.
SVG is a declarative specification, similar to HTML or XML, which produces crisp vector
graphics with a limited set of supported interactivity through hover and transformation events.
Canvas2D is an imperative way of rendering to a scene by using JavaScript.
Instead of relying on declaratively positioned nodes, the scene is rendered iteratively
through a sequence of draw calls, akin to physically moving a pen across a physical paper canvas.

\section{Two- and Three-Dimensional Web Graphics}
For more complex two-dimensional, or even three-dimensional scenes,
the capabilities of the GPU can be utilized through the means of WebGL or WebGPU.
Both WebGL and WebGPU are actively developed and maintained by the Khronos Group, a consortium
of leading technology companies, focussed on the development and maintenance of
interoperability standards for three-dimensional graphics, augmented reality standards and
machine learning \parencite{khronos-web}.

\subsection{WebGL}
WebGL is an abstraction layer (wrapper) over the OpenGL graphics library.
This enables the usage of high performance OpenGL shader code with JavaScript.
It is natively supported by all major browsers and has remained the current standard for
complex 3D rendering.

Many existing programs work natively with WebGL, creating a large ecosystem of usable libraries
and tools that either utilize WebGL directly or allow export to WebGL code.
The project started in 2009, based on the OpenGL ES 3.0 specification.
Development slowed down in 2017, as development funding largely shifted towards WebGPU.
One of the drawbacks of WebGL is that lacking support for compute shaders, a set of shaders
specifically useful for machine learning and matrix manipulation tasks.

\subsection{WebGPU}
WebGPU is the natural successor to the WebGL specification.
With the support of major companies like Google and Mozilla, it went through steady
development since 2017, and is expected to release in quarter four of 2022.
WebGPU is an abstraction layer that drives Vulkan, Metal or DirectX 12 natively.
The GPU hardware is accessed indirectly over this abstraction layer.
This approach allows developers to target multiple platforms with a single set of source code.

Since the specification is still in development, no major browsers natively support WebGPU
out of the box. All browsers support the specification after enabling an unstable feature flag.
A broader audience is targeted through current Google Chrome Origin Trials.



\cleardoublepage
%----------------------------------------------------------------
%
%  File    :  fundamentals.tex
%
%  Authors :  Pinheiro de Souza, Schintler, Steinkellner
% 
%  Created :  22 Nov 2022
% 
%----------------------------------------------------------------

\chapter{Fundamentals of Web Graphics}

\label{chap:Fundamentals}

Before diving into WebGPU example code, some fundamental concepts must be introduced.
This section covers the essential stages needed for a minimal example.

\section{Pipelines}
The WebGPU specification lays out GPU commands sequentially, with similarities to WebGL.
It supports two different pipelines:
The \textbf{compute pipeline} is responsible for executing parallel
computations on the GPU hardware. It only consists of a single programmable
stage, customized via the compute shader.
While providing numerous benefits to parallel computations over CPU-executed code, for the sake of brevity, it won't be covered within this survey.
This document mainly focuses on the \textbf{render pipeline}, which houses both the \textbf{vertex shader} and the \textbf{fragment shader}.

\section{Shaders}
Shaders are programmable stages within GPU pipelines.
There are two programmable stages within the render pipeline:
The vertex shader and the fragment shader.

\subsection{Vertex Shader}
Vertex shaders are responsible for defining all vertices of a desired primitive.
Each vertex has a position and a set of attributes associated with it.
Vertex coordinates are relative to the centre of the primitive, and must be defined per primitive.
Additional colour values, required by the fragment shader in a later stage of the pipeline, must be appended here to allow pass-through.
Image \ref{fig:vertex-01} illustrates simple example of a vertex shader for a triangular primitive.

\begin{figure}[tp]
\centering
\includegraphics[keepaspectratio,width=\linewidth,height=\halfh]
{images/vertex-01.pdf}

\caption[Vertex shader, example illustration]
{
  Exemplary illustration of a vertex shader for a primitive triangle.
  The vertices V1, V2, V3 define the vertices of the primitive.
\imgcredit{Created by the authors.}
}
\label{fig:vertex-01}
\end{figure}

\subsection{Intermediate Step}
Following the vertex shader, the render pipeline contains an intermediate step that cannot be modified directly.
First, the vertices are assembled into primitives.
These primitives are subsequently clipped to fit within the bounding box of the viewport, before entering the rasterization stage.
This rasterization stage transforms the primitive into rasterized points on the screen.

\subsection{Rasterization}
The rasterization stage is not directly programmable like the vertex or fragment shaders,
but it is a vital part of the rendering pipeline. During this stage, the vertex information
is transformed into rasterized segments on the screen, corresponding to the available pixels in the
viewport. Additionally, this stage includes the culling of obstructed polygons. Any front-facing
polygons are rendered and evaluated, whilst all obstructed ones are discarded.

The most performant way of rasterization is evaluating each available pixel at its center.
If that center point is inside a primitive, the corresponding pixel should be coloured.
Whilst being performant and fairly simple to execute, this leads to a stair-like effect on edges that
should be smooth, often called "jaggies".
Figure \ref{fig:rasterization} shows the naive implementation of the rasterization process.
The rasterized points are shaded blue, all discarded pixels are white.

\begin{figure}[tp]
\centering
\includegraphics[keepaspectratio,width=\linewidth,height=\halfh]
{images/rasterization.pdf}

\caption[Rasterization, example illustration]
{
  Exemplary illustration of the rasterization step for a primitive triangle.
  Rasterized points inside the primitive are coloured in blue, whilst points outside are displayed white.
\imgcredit{Created by the authors.}
}
\label{fig:rasterization}
\end{figure}

For most applications, it is desirable to reduce the stair-like jaggies that the rasterization process produces.
This is often done through a process called Anti-Aliasing.
WebGPU provides a built-in way to deal with aliasing by enabling a technique called multisampling for the
rasterization process. This is done through optional configuration during the setup of the scene.

With multisampling enabled, each pixel is evaluated on multiple points, instead of just the centre.
The sample points are deliberately placed near the edge, and are grouped in a sample mask.
The final pixel value is the result of interpolating between all four point samples of the sample mask.

Figure \ref{fig:multisampling} illustrates the process of multisampling on an example triangle.

\begin{figure}[tp]
\centering
\includegraphics[keepaspectratio,width=\linewidth,height=\halfh]
{images/multisampling.pdf}

\caption[Multisampling, example illustration]
{
  Exemplary illustration of multisampling during the rasterization step for a primitive triangle.
  Sample points are indicated with red circles for two of the pixels. All pixels are evaluated with the same
  sample mask, and the final pixel colour value is the result of interpolating between all sample points.
\imgcredit{Created by the authors.}
}
\label{fig:multisampling}
\end{figure}


\subsection{Fragment Shader}
The vertex shader provides vertices that define the shape of a primitive.
Within the fragment shader, this primitive is coloured.
The vertices enter this stage with a defined colour value, and the fragment shader interpolates between these values.
It produces one fragment per rasterization point, and its execution is parallel.
Figure \ref{fig:fragment-01} shows the desired output of this stage, whilst also highlighting the biggest challenge: Rasterization.

\begin{figure}[tp]
\centering
\includegraphics[keepaspectratio,width=\linewidth,height=\halfh]
{images/fragment-01.pdf}

\caption[Fragment shader, example illustration]
{
  Exemplary illustration of a fragment shader for a primitive triangle to showcase the desired output.
  Each vertex is defined by its position and an associated colour value.
  For a uniformly coloured triangle, all three colour values must be equal.
\imgcredit{Created by the authors.}
}
\label{fig:fragment-01}
\end{figure}


\cleardoublepage
%----------------------------------------------------------------
%
%  File    :  practical-example.tex
%
%  Authors :  Pinheiro de Souza, Schintler, Steinkellner
% 
%  Created :  22 Nov 2022
% 
%----------------------------------------------------------------

\chapter{Practical Example of using WebGPU}

\label{chap:PracticalExample}

Due to the nature of rendering directly with a GPU using WebGPU is a rather tedious process as the essential code to render a scene needs to be executed on the GPU.
This section covers the general program flow and the essential steps needed for a minimal working example. Due to WebGPU's complexity, not all details of the implementation are explained in detail. 


\section{General Flow}

The inherent complexity of programming fast Web Graphics via WebGPU stems from the way one has to interact with a GPU itself.
In contrast to simpler 2D graphics like SVG and Canvas2D WebGPU is just an abstraction layer on top of broadly used GPU APIs like Vulkan.
It, therefore, has to perform similar steps as native software rendering 3D scenes. As can be seen in figure \ref*{fig:webgpu-explain} we first need to
collect all information of a scene, like Vertex positions and color, as well as the matching WGSL Shader Code and mangle it via the CPU into a properly structured
buffer on the GPU's V-RAM. Only then can the CPU hand over control and allow the GPU to start execution of the WGSL Shader Code which will use the buffer to render a proper scene.
This workflow is described in more detail in the following sections.


\begin{figure}[tp]
  \centering
  \includegraphics[keepaspectratio,width=\linewidth,height=\halfh]
  {images/wgpu-explain.pdf}

  \caption[Dataflow in WebGPU Example]
  {
    An exemplary illustration of how instructions and data of a 3D scene have to be handled
    to use WebGPU.
    \imgcredit{Created by the authors.}
  }
  \label{fig:webgpu-explain}
\end{figure}


\section{Steps}

To successfully render a 3D scene using WebGPU several different steps are required. 
This section covers the essential steps needed for a minimal working example.



\subsection{Step 1 - Encoding the Vertex Information}
\label{section:practical-step-1}

First of all, as mentioned above, the whole scene needs to be encoded into a stream of numbers. WebGPU does not offer any form of scene layout, meaning a developer
has to come up with his structure for storing a scene's information about the location of objects, vertices and their colors. As can be seen in listing \ref*{code:vertex-encode} a general 
purpose \emph{Vertex} class is created to hold all the necessary information required to form a basic scene. It exposes a  \emph{encode()} function to easily merge all properties of a single vertex 
into its numeric values. The additional function \emph{encodeVertices()} is designed to then combine multiple vertices into a single stream of numeric values. 

\begin{samepage}
  \lstinputlisting[
    float=tp,
    xleftmargin=0cm,              % no extra margins for floats
    xrightmargin=0cm,             % no extra margins for floats
    basicstyle=\footnotesize\ttfamily,
    frame=shadowbox,
    numbers=left,
    caption={[Code Snippet: Vertex Encoding]
    {
      An exemplary illustration of how to encode vertex information for use in WebGPU
      \imgcredit{Created by the authors.}
    }},
    language=TypeScript,
    firstnumber=1,
    label=code:vertex-encode
    ]
    {listings/vertex.ts}
\end{samepage}


\subsection{Step 2 - Creating a Buffer}

The encoded data created in listing \ref*{section:practical-step-1} then needs to be stored in the GPU itself. To accomplish this a buffer needs to be created, as can be seen in listing \ref*{code:create-buffer}. 
WebGPU exposes a \emph{createBuffer()} function on each \emph{GPUDevice} which allows interaction with the GPU's memory. Buffer sizes need to align to 4-byte steps and each buffer needs to 
have a designated usage, resembling in which stage the buffer is used (vertex stage or fragment stage). After writing the data to the buffer it is important to also \emph{unmap()} the buffer 
to hand over control to the GPU. 

\begin{samepage}
  \lstinputlisting[
    float=tp,
    xleftmargin=0cm,              % no extra margins for floats
    xrightmargin=0cm,             % no extra margins for floats
    basicstyle=\footnotesize\ttfamily,
    frame=shadowbox,
    numbers=left,
    caption={[Code Snippet: \emph{createBuffer}]
    {
      An exemplary illustration of how instructions and data of a 3D scene have to be handled
      to use WebGPU.
      \imgcredit{Created by the authors.}
    }},
    language=TypeScript,
    firstnumber=1,
    label=code:create-buffer
    ]
    {listings/helper.ts}
\end{samepage}


\subsection{Step 3 - WGSL Shader Code}
\label{section:shader-code}

Even after the data is encoded and written on the GPU RAM one still needs to instruct the GPU how the data is actually to be processed. This is accomplished by writing custom
shader code which is executed on the GPU in parallel. A minimal example can be seen in listing \ref*{code:shader-code}.  It contains 2 main functions for the vertex and fragment stages respectively. 
The function \emph{vertex\_main} is executed once per vertex and is responsible for setting an appropriate position and color per vertex. Afterwards, the function \emph{fragment\_main} is 
executed in parallel for each rasterized pixel of the scene to define the actual color value of each pixel. 

\begin{samepage}
  \lstinputlisting[
    float=tp,
    xleftmargin=0cm,              % no extra margins for floats
    xrightmargin=0cm,             % no extra margins for floats
    basicstyle=\footnotesize\ttfamily,
    frame=shadowbox,
    numbers=left,
    caption={[Code Snippet: WebGPU Shader Code]
    {
      An exemplary code snippet of how to write WGSL Shader Code
      \imgcredit{Created by the authors.}
    }},
    language=WGSL,
    firstnumber=1,
    label=code:shader-code
    ]
    {listings/shader.wgsl}
\end{samepage}


\subsection{Step 4 - Setting up the Final Pipeline}

With all the previously mentioned parts in place, one can now set up the actual pipeline. As can be seen in listing \ref*{code:pipeline-setup} one first needs to access a \emph{HTMLCanvasElement}. This 
then exposes a \emph{WebGPUConvasContext}. This context then allows the request of an adapter which in turn provides access to a \emph{GPUDevice}.
This device can then be used to create a buffer with the encoded vertices data. Afterward, a render pipeline is created by supplying the necessary information for both the vertex and the fragment stage.
Within this step, the shader code of listing \ref*{section:shader-code} is passed along as well. At last, a \emph{CommandEncoder} is used to encode a GPU command to render an actual image from the pipeline to the respective canvas on the screen. 

\begin{samepage}
  \lstinputlisting[
    float=tp,
    xleftmargin=0cm,              % no extra margins for floats
    xrightmargin=0cm,             % no extra margins for floats
    basicstyle=\footnotesize\ttfamily,
    frame=shadowbox,
    numbers=left,
    caption={[Code Snippet: WebGPU Pipeline]
    {
      An exemplary illustration of how to set up a WebGPU pipeline
      \imgcredit{Created by the authors.}
    }},
    language=TypeScript,
    firstnumber=11,
    label=code:pipeline-setup
    ]
    {listings/main.ts}
\end{samepage}

\cleardoublepage
%----------------------------------------------------------------
%
%  File    :  libraries.tex
%
%  Authors :  Pinheiro de Souza, Schintler, Steinkellner
% 
%  Created :  22 Nov 2022
% 
%----------------------------------------------------------------

\newcommand*{\codeThreejs}[1]{
  \inputminted[
    frame=lines,
    framesep=2mm,
    baselinestretch=1.2,
    linenos,
    fontsize=\scriptsize,
    breaklines=true,
    highlightlines={30-39,43-49,81-84},
    firstline=30,
    lastline=84
  ]{typescript}{#1}}

\newcommand*{\codeBabylon}[1]{
  \inputminted[
    frame=lines,
    framesep=2mm,
    baselinestretch=1.2,
    linenos,
    fontsize=\scriptsize,
    breaklines=true,
    highlightlines={21-27,39-45},
    firstline=21,
    lastline=45
  ]{typescript}{#1}}

\chapter{Libraries}

\label{chap:Libraries}

Getting started with WebGL is quite simple. 
One factor that makes it that simple are the many libraries that exist and provide an interface for WebGL.
However the support of most libraries for WebGPU is still lacking even though we should expect a release this quarter.

\section{Two.js}

Two.js is one of the most prominent library for working in 2D, hence the name.
Providing a drawing api for drawing in multiple contexts, e.g. svg, canvas and WebGL.\\
On the forum of Two.js the developers provide a little inside into the development.
Talking about the next steps of the library and if WebGPU will be supported.
Stating that \begin{displayquote} "... a WebGPU renderer seems like the natural evolution of renderers in Two.js." \end{displayquote}\parencite{two_quote}.
However also pointing to the fact that WebGPU is not released yet and thus not starting to work on it until there is an official release.

\section{Three.js}

Three.js is as the name suggests a 3D JavaScript library. 
What Two.js is in the 2D domain in the 3D domain there is Three.js.\\
In contrast to Two.js, Three.js is already actively working on implementing a WebGPU renderer.
Currently there is already a WebGPU renderer implemented in Three.js that is usable. 
Even though there are still some issues and not all of the features work as intended, it is still usable to a certain degree.
Which is also visible by the several different examples that are already online \parencite{three_examples}.
As well as in our own example that we created using Three.js and the WebGPU renderer. \\

Listing \ref{code:three-scene} shows the required setup for a simple scene in Three.js. \\
First, one has to create each object invidually. Afterwards, they can be added to the scene.

\begin{listing}
  \centering
  \codeThreejs{../threejs-example/src/main.ts}
  \caption[Code Snippet: Three.js Example]
  {
    Scene, camera and WebGPU setup in Three.js, followed by simple object creation.
    \imgcredit{Created by the authors.}
  }
  \label{code:three-scene}
\end{listing}


After adding hover interactivity it is possible to get create a scene that reacts to the position of the mouse cursor.
The initial state of the scene can be seen in figure \ref{fig:three_img1}. The results of the hover effect are visible in figure \ref{fig:three_img2}.

\begin{figure}[tp]
  \centering
  \includegraphics[keepaspectratio,width=\linewidth,height=\halfh]
  {images/three_example_img1.png}
  
  \caption[Three.js Example Without Mouse Interactivity]
  {
  A simple Three.js example without the mouse hovering over any object.
  \imgcredit{Created by the authors.}
  }
  \label{fig:three_img1}
\end{figure}

\begin{figure}[tp]
  \centering
  \includegraphics[keepaspectratio,width=\linewidth,height=\halfh]
  {images/three_example_img2.png}
  
  \caption[Three.js Example With Mouse Interactivity]
  {
  A simple Three.js example with the mouse hovering over multiple objects.
  \imgcredit{Created by the authors.}
  }
  \label{fig:three_img2}
\end{figure}

\section{Babylon.js}

Last but not least there is Babylon.js.
Babylon.js started the integration process for WebGPU already back in 2019 \parencite{babylon_start_webgpu}. 
The first usable WebGPU renderer in Babylon.js was released with version 5.0 of the library \parencite{babylon_released}.
Another milestone that Babylon.js was able to achieve was to got feature parity with the WebGL renderer back in January 2022 \parencite{babylon_parity}.
Thus providing full WebGPU support for the library.

Similar to Three.js it is possible to create an example scene quite easily.
First, one has to create a scene and specify the engine one wants to use.
After this we can start creating and adding objects to the scene, as shown in listing \ref{code:babylon-scene}.

\begin{listing}
  \centering
  \codesnippet{19}{55}{typescript}{../babylon-js-example/src/main.ts}
  \caption[Code Snippet: Babylon Example]
  {
    Scene, camera and WebGPU setup in Babylon.js, followed by simple object creation.
    \imgcredit{Created by the authors.}
  }
  \label{code:babylon-scene}
\end{listing}

Figure \ref{fig:babylon_example_img1} shows an interactive example of a Babylon.js scene where the user can drag and drop
a cube.
\begin{figure}[tp]
  \centering
  \includegraphics[keepaspectratio,width=\linewidth,height=\halfh]
  {images/babylon_example_img1.png}
  
  \caption[Babylon.js Example With Mouse Interactivity]
  {
  Simple interactive example of a Babylon.js scene where the user can drag and drop a cube.
  \imgcredit{Created by the authors.}
  }
  \label{fig:babylon_example_img1}
\end{figure}



\cleardoublepage
%----------------------------------------------------------------
%
%  File    :  cross-platform.tex
%
%  Authors :  Pinheiro de Souza, Schintler, Steinkellner
% 
%  Created :  22 Nov 2022
% 
%----------------------------------------------------------------

\chapter{Cross-Platform}

\label{chap:Cross-Platform}

\section{WebAssembly}
Instead of writing strictly platform-dependent native code, WebAssembly can be used to target
execution within the context of the browser \parencite{wasm}. It offers a load-time-efficient virtual stack
machine, executed within the sandbox of a browser tab. 
It serves as a compile target for numerous languages, such as C, C++, or Rust \parencite{ivis-2022}.
Compilation ends with a single executable file with the \lstinline{*.wasm} format,
which can be loaded and executed within the browser.

Most modern browsers already support the use of WebAssembly, with a 95\% coverage across
global internet users, as shown in figure \ref{fig:wasm}.

\begin{figure}[tp]
\centering
\includegraphics[keepaspectratio,width=\linewidth,height=\halfh]
{images/wasm.png}

\caption[Browser Support of WebAssembly]{
  WebAssembly is supported for about 95\% of global internet users.
\imgcredit{Screenshot taken by the authors of this paper.}
}
\label{fig:wasm}
\end{figure}



\cleardoublepage
%----------------------------------------------------------------
%
%  File    :  rust.tex
%
%  Authors :  Pinheiro de Souza, Schintler, Steinkellner
% 
%  Created :  22 Nov 2022
% 
%----------------------------------------------------------------


\chapter{Rust}

\label{chap:Rust}

Since the WebGPU Shading Language (WGSL) is largely a syntactic mix between Rust and C,
working with WebGPU in Rust seems to be a natural fit.
Over the course of this project, we have explored the combination of Rust and WebGPU by implementing a simple example.

Rust provides a very easy way to compile code into WebAssembly through a package called \emph{wasm-bindgen}.
The package \emph{wasm-bindgen-futures} is needed to wait for asynchronous operations within the browser.
The following example was created by following the excellent tutorial on \textcite{rust-wgpu}.


\section{Steps}

% --------------------------------------------------------------------------- %
% Section: Repository Setup

\subsection{Step 1 --- Repository Setup}
WebGPU is provided to rust via the \emph{wgpu} package.
Listing \ref{code:cargo} shows the \emph{Cargo.toml} file used in our minimal example.

\begin{samepage}
  \lstinputlisting[
    float=tp,
    xleftmargin=0cm,              % no extra margins for floats
    xrightmargin=0cm,             % no extra margins for floats
    basicstyle=\footnotesize\ttfamily,
    frame=shadowbox,
    numbers=left,
    caption={[Code Snippet: Cargo.toml]
    {
      A simple \emph{Cargo.toml} file for WebAssembly and WebGPU in Rust.
      \imgcredit{Created by the authors.}
    }},
    firstnumber=1,
    label=code:cargo
    ]
    {listings/wasm/Cargo.toml}
\end{samepage}


Once packages are properly setup and a module bundler like Vite or Webpack is configured to handle WASM files properly,
implementation of the WebGPU code is fairly straightforward.


% --------------------------------------------------------------------------- %
% Section: Creating the Canvas

\subsection{Step 2 --- Creating the Canvas}
Similar to the previous TypeScript example, the first step is always window and device acquisition, required in order
to set up a canvas and a GPU handle to draw to. In Rust, this is done through requesting an HTML div element from the browser.
Afterwards, a new canvas element for WebGPU is pushed to the DOM, as demonstrated in listing \ref{code:rust-canvas}.

\begin{samepage}
  \lstinputlisting[
    float=tp,
    xleftmargin=0cm,              % no extra margins for floats
    xrightmargin=0cm,             % no extra margins for floats
    basicstyle=\footnotesize\ttfamily,
    frame=shadowbox,
    numbers=left,
    caption={[Code Snippet: Canvas Setup]
    {
      Example of setting up a canvas element inside the HTML DOM via Rust.
      \imgcredit{Created by the authors.}
    }},
    language=Rust,
    firstnumber=165,
    label=code:rust-canvas
    ]
    {listings/wasm/lib_canvas.rs}
\end{samepage}


% --------------------------------------------------------------------------- %
% Section: GPU Handle Acquisition

\subsection{Step 3 --- GPU Handle Acquisition}
The following stages start with initializing the WebGPU backend, followed by surface and adapter creation and acquisition
of the GPU handle, as demonstrated in listing \ref{code:rust-handle}.

\begin{samepage}
  \lstinputlisting[
    float=tp,
    xleftmargin=0cm,              % no extra margins for floats
    xrightmargin=0cm,             % no extra margins for floats
    basicstyle=\footnotesize\ttfamily,
    frame=shadowbox,
    numbers=left,
    caption={[Code Snippet: Backend, surface, adapter, and GPU handle setup]
    {
      Example code of setting up the initial backend for WebGPU, followed by surface creation on the
      specified window. Afterwards, an adapter is created and the handle to the GPU is requested.
      \imgcredit{Created by the authors.}
    }},
    language=Rust,
    firstnumber=23,
    label=code:rust-handle
    ]
    {listings/wasm/lib_handle.rs}
\end{samepage}

% --------------------------------------------------------------------------- %
% Section: Shader and Viewport Configuration %

\subsection{Step 4 --- Shader Generation, Viewport Configuration}
Shader generation is done by including a \emph{*.wgsl} file according to the WebGPU specification.
After setting up the surface configuration with the appropriate viewport size and texture format,
the pipeline can be created. The shader code can be examined in listing \ref{code:rust-wgsl}.
The required Rust code to include the shader is shown in listing \ref{code:rust-shader}.


\begin{samepage}
  \lstinputlisting[
    float=tp,
    xleftmargin=0cm,              % no extra margins for floats
    xrightmargin=0cm,             % no extra margins for floats
    basicstyle=\footnotesize\ttfamily,
    frame=shadowbox,
    numbers=left,
    caption={[Code Snippet: WGSL Shader Code]
    {
      Example code of a simple triangular vertex and fragment shader pair.
      \imgcredit{Created by the authors.}
    }},
    language=WGSL,
    firstnumber=1,
    label=code:rust-wgsl
    ]
    {listings/wasm/shader.wgsl}
\end{samepage}

\begin{samepage}
  \lstinputlisting[
    float=tp,
    xleftmargin=0cm,              % no extra margins for floats
    xrightmargin=0cm,             % no extra margins for floats
    basicstyle=\footnotesize\ttfamily,
    frame=shadowbox,
    numbers=left,
    caption={[Code Snippet: Shader and Viewport Setup]
    {
      Example code of accessing the WebGPU shader inside our Rust example and configuring the viewport.
      \imgcredit{Created by the authors.}
    }},
    language=Rust,
    firstnumber=53,
    label=code:rust-shader
    ]
    {listings/wasm/lib_shader.rs}
\end{samepage}


% --------------------------------------------------------------------------- %
% Section: Render Function

\subsection{Step 5 --- Render Function}
Finally, the simple example is completed by setting up a render function that is called for every frame,
as demonstrated in listing \ref{code:rust-render}.

\begin{samepage}
  \lstinputlisting[
    float=tp,
    xleftmargin=0cm,              % no extra margins for floats
    xrightmargin=0cm,             % no extra margins for floats
    basicstyle=\footnotesize\ttfamily,
    frame=shadowbox,
    numbers=left,
    caption={[Code Snippet: Event Loop Setup]
    {
      Example code of setting up an event loop to redraw on every frame in our Rust example.
      \imgcredit{Created by the authors.}
    }},
    language=Rust,
    firstnumber=181,
    label=code:rust-render
    ]
    {listings/wasm/lib_render.rs}
\end{samepage}


\cleardoublepage
%----------------------------------------------------------------
%
%  File    :  performance.tex
%
%  Authors :  Pinheiro de Souza, Schintler, Steinkellner
% 
%  Created :  22 Nov 2022
% 
%----------------------------------------------------------------

\chapter{Performance}
\label{chap:Performance}

Performance is a crucial aspect of modern web development. Currently, WebGL is used for fast web graphics. WebGPU aims to improve performance in multiple ways. 
The following sections explore and compare performance differences between WebGL and WebGPU. 

\section{Babylon.js}

Since Babylon has already support for both renderers it also provides a benchmark for both.
By rendering the same scene once with WebGL \parencite{babylon_example_webgl} and once with WebGPU \parencite{babylon_example_webgpu}.
Providing also a side-by-side comparison via a video.
There it is visible that WebGPU gives a performance boost in comparison with WebGL.
Showcased by the lower CPU load as well as the higher frames per second (FPS).

\section{Water Simulation}

Another graphics comparison was published by Eytan Manor on the medium webpage \parencite{water_sim_perf}.
Comparing OpenGL and Vulkan the underlying APIs of WebGL and WebGPU. 
In this benchmark, a water mesh gets rendered and transformed to simulate realistic movements.
The results of this benchmark are visible in \ref{fig:water_sim_gpu}.

\begin{figure}[tp]
  \centering
  \includegraphics[keepaspectratio,width=\linewidth,height=\halfh]
  {images/water_sim_fps.png}
  
  \caption[Water Simulation Results --- Average FPS]
  {
  Vulkan to OpenGL benchmark result for Average FPS. 
  \imgcredit{Image taken from: https://eytanmanor.medium.com/the-story-of-webgpu-the-successor-to-webgl-bf5f74bc036a and used under § 42f.(1) of Austrian copyright law}
  }
  \label{fig:water_sim_fps}
\end{figure}

\begin{figure}[tp]
  \centering
  \includegraphics[keepaspectratio,width=\linewidth,height=\halfh]
  {images/water_sim_cpu.png}
  
  \caption[Water Simulation Results --- Average CPU Load]
  {
  Vulkan to OpenGL benchmark result for Average CPU load in \%. 
  \imgcredit{Image taken from: https://eytanmanor.medium.com/the-story-of-webgpu-the-successor-to-webgl-bf5f74bc036a and used under § 42f.(1) of Austrian copyright law}
  }
  \label{fig:water_sim_cpu}
\end{figure}

\begin{figure}[tp]
  \centering
  \includegraphics[keepaspectratio,width=\linewidth,height=\halfh]
  {images/water_sim_gpu.png}
  
  \caption[Water Simulation Results --- Average GPU Load]
  {
  Vulkan to OpenGL benchmark result for Average GPU load in \%. 
  \imgcredit{Image taken from: https://eytanmanor.medium.com/the-story-of-webgpu-the-successor-to-webgl-bf5f74bc036a and used under § 42f.(1) of Austrian copyright law}
  }
  \label{fig:water_sim_gpu}
\end{figure}

These results show it quite well that Vulkan and thus WebGPU give us a great performance boost at a much lower CPU cost.
In contrast to OpenGL, Vulkan offers a low-level API that allows developers to parallelize workloads.
It provides higher performance, visible in the FPS, and more balanced CPU to GPU usage.

\section{Matrix Multiplication}

One benchmark that also should not be omitted is a comparison of computation tasks
since WebGPU is equipped with a compute pipeline and not only with a render pipeline.\\
In WebGL, there is not something similar to WebGPUs compute pipeline.
However, it is possible to mimic the compute pipeline with the help of a little trick.
The data used for computations need to be converted into an image.
This image is then passed to the pipeline and the fragment shader, where the actual computations are performed.
After the computations are done the results are stored in the same image as pixel color values.
Those values now have to be read from the image back into the original data format. \\
In contrast, WebGPU provides compute shaders for those tasks.
The data is passed to the GPU with a buffer and the computations are done asynchronously. 
Additionally, there is no need to convert or extract the data from an image, giving another performance boost.

Such a benchmark is visible in \ref{fig:matrix_mult}. 
In this benchmark, the size of the matrix gets continuously bigger. It demonstrates that the "Image hack" with WebGL is approximately 3.5 times slower than the compute shaders of WebGPU.
Additionally, it can process more data and it does not block the main thread.


\begin{figure}[tp]
  \centering
  \includegraphics[keepaspectratio,width=\linewidth,height=\halfh]
  {images/matrix_mult.png}
  
  \caption[Matrix Multiplication Benchmark]
  {
  \imgcredit{Image taken from: https://pixelscommander.com/javascript/webgpu-computations-performance-in-comparison-to-webgl/ and used under § 42f.(1) of Austrian copyright law}
  }
  \label{fig:matrix_mult}
\end{figure}



\nocite{*}

\cleardoublepage
% for now, switch to language english
% hack to force unix date for biblio, biblatex 3.11
\begin{otherlanguage}{english}
\printbibliography[heading=bibintoc]
\end{otherlanguage}


\end{document}

