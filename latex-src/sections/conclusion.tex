%----------------------------------------------------------------
%
%  File    :  conclusion.tex
%
%  Authors :  Pinheiro de Souza, Schintler, Steinkellner
% 
%  Created :  22 Nov 2022
% 
%----------------------------------------------------------------

\chapter{Concluding Remarks}


While WebGPU feels a little more modern than WebGL in its syntax,
writing native WebGPU code still remains to be a complicated task. This
is partly due to the fact that GPU compute units have to work in
parallel without blocking, which requires a different program flow in
comparison to traditional CPU-based programming.
%
Although the WebGPU specification itself appears to be fairly complete,
there are still some undocumented options that have to be tested through
trial and error. Support in libraries and tools is growing, but still
severely lacking in terms of development friendliness through static
type checking or fully featured documentation.
%
According to our research, the best two libraries for WebGPU are
currently Babylon.js and Three.js, but both suffer from the
aforementioned problems.

While WebGPU will most likely replace WebGL in the long run, we would
currently discourage from targeting it directly. It is advised to use a
graphics library like Babylon.js or Three.js, which both feature
settings to target either WebGL or WebGPU. It should be easy to port an
existing application written with these Tools from WebGL over to WebGPU,
once the standard has matured properly and browser support reaches a
broader audience.
