%----------------------------------------------------------------
%
%  File    :  intro.tex
%
%  Authors :  Thomas Pinheiro de Souza, Schintler, Steinkellner
% 
%  Created :  22 Nov 2022
% 
%----------------------------------------------------------------


\chapter{Introduction}

\label{chap:Intro}
Web graphics are a way to include visual scenes on websites or web applications.
While static images can be easily included to an existing site by using the HTML image tag,
interactivity is often a crucially important feature needed to bring ideas or scenes to life.
It can have a positive impact on overall site engagement \parencite{web-engagement-literature}.
Interactivity can be a very broad term to define, but it includes anything a user can do within
a graphic to manipulate the scene. The traditional two-dimensional ways of including interactive
web graphics ot a website are SVG and Canvas2D.
SVG is a declarative specification, similar to HTML or XML, producing crisp vector art with
limited options of interactivity through hover and transformation events.
Canvas2D is an imperative way of rendering to a scene by using JavaScript. TODO EXTEND?

For more complex two-dimensional, or three-dimensional scenes, the capabilities of the GPU can be utilized
through the means of WebGL or WebGPU.


% \begin{figure}[tp]
% \centering
% \includegraphics[keepaspectratio,width=\linewidth,height=\halfh]
% {images/?????.xxx}
% 
% \caption[Self-explanatory caption for this image]
% {
% Self-explanatory text to describe this image.
% \imgcredit{We got this image from \textcite{BIBTEX_KEY}. Used under § 42f.(1) of Austrian copyright law.}
% }
% \label{fig:my-awesome-fig}
% \end{figure}

\section{Section}
Sections look like this.

\subsection{Subsection}
Subsections look like this.

\begin{itemize}
\item This is a bulleted list
\item Here's another entry
\end{itemize}
