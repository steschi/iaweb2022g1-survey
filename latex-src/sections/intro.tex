%----------------------------------------------------------------
%
%  File    :  intro.tex
%
%  Authors :  Pinheiro de Souza, Schintler, Steinkellner
% 
%  Created :  22 Nov 2022
% 
%----------------------------------------------------------------


\chapter{Introduction}

\label{chap:Intro}
Web graphics are a way to include visual scenes on websites or web applications.
While static images can be easily included to an existing site by using the HTML image tag,
interactivity is often a crucially important feature needed to bring ideas or scenes to life.
It can have a positive impact on overall site engagement \parencite{web-engagement-literature}.
Interactivity can be a very broad term to define, but it includes anything a user can do within
a graphic to manipulate the scene.

\section{Two-Dimensional Web Graphics}
The traditional two-dimensional approaches of including interactive
web graphics to a website use either SVG or Canvas2D.
SVG is a declarative specification, similar to HTML or XML, which produces crisp vector
graphics with a limited set of supported interactivity through hover and transformation events.
Canvas2D is an imperative way of rendering to a scene by using JavaScript.
Instead of relying on declaratively positioned nodes, the scene is rendered iteratively
through a sequence of draw calls, akin to physically moving a pen across a physical paper canvas.

\section{Two- and Three-Dimensional Web Graphics}
For more complex two-dimensional, or even three-dimensional scenes,
the capabilities of the GPU can be utilized through the means of WebGL or WebGPU.
Both WebGL and WebGPU are actively developed and maintained by the Khronos Group, a consortium
of leading technology companies, focussed on the development and maintenance of
interoperability standards for three-dimensional graphics, augmented reality standards and
machine learning \parencite{khronos-web}.

\subsection{WebGL}
WebGL is an abstraction layer (wrapper) over the OpenGL graphics library.
This enables the usage of high performance OpenGL shader code with JavaScript.
It is natively supported by all major browsers and has remained the current standard for
complex 3D rendering.

Many existing programs work natively with WebGL, creating a large ecosystem of usable libraries
and tools that either utilize WebGL directly or allow export to WebGL code.
The project started in 2009, based on the OpenGL ES 3.0 specification.
Development slowed down in 2017, as development funding largely shifted towards WebGPU.
One of the drawbacks of WebGL is that lacking support for compute shaders, a set of shaders
specifically useful for machine learning and matrix manipulation tasks.

\subsection{WebGPU}
WebGPU is the natural successor to the WebGL specification.
With the support of major companies like Google and Mozilla, it has undgerone steady
development since 2017, and is expected to release in quarter four of 2022.
WebGPU is an abstraction layer that drives Vulkan, Metal or DirectX 12 natively.
The GPU hardware is accessed indirectly over this abstraction layer.
This approach allows developers to target multiple platforms with a single set of source code.

Since the specification is still in development, no major browsers natively support WebGPU
out of the box. All browsers support the specification after enabling an unstable feature flag.
A broader audience is targeted through current Google Chrome Origin Trials.

