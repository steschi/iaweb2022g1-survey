%----------------------------------------------------------------
%
%  File    :  rust.tex
%
%  Authors :  Pinheiro de Souza, Schintler, Steinkellner
%
%  Created :  22 Nov 2022
%
%----------------------------------------------------------------


\chapter{Rust}

\label{chap:Rust}

Since the WebGPU Shading Language (WGSL) is largely a syntactic mix
between Rust and C, working with WebGPU in Rust seems to be a natural
fit. Over the course of this project, we have explored the combination
of Rust and WebGPU by implementing a simple example.
%
Rust provides a very easy way to compile code into WebAssembly through a
package called \emph{wasm-bindgen}. The package
\emph{wasm-bindgen-futures} is needed to wait for asynchronous
operations within the browser. The following example was created by
following the excellent tutorial on \textcite{rust-wgpu}.


% --------------------------------------------------------------------------- %
% Section: Repository Setup

\section{Step 1 --- Repository Setup}
WebGPU is provided to rust via the \emph{wgpu} package. Listing
\ref{code:cargo} shows the \emph{Cargo.toml} file used in our minimal
example.
%
Once packages are properly setup and a module bundler like Vite or
Webpack is configured to handle WASM files properly, implementation of
the WebGPU code is fairly straightforward.

\begin{samepage}
  \lstinputlisting[ float=tp, xleftmargin=0cm,
    % no extra margins for floats
    xrightmargin=0cm,             % no extra margins for floats
    basicstyle=\footnotesize\ttfamily, frame=shadowbox, numbers=left,
    caption={[Code Snippet: Cargo.toml] { A simple \emph{Cargo.toml}
    file for WebAssembly and WebGPU in Rust.
      \imgcredit{Created by the authors.}
    }}, firstnumber=1, label=code:cargo ] {listings/wasm/Cargo.toml}
\end{samepage}



% --------------------------------------------------------------------------- %
% Section: Creating the Canvas

\section{Step 2 --- Creating the Canvas}
Similar to the previous TypeScript example, the first step is always
window and device acquisition, required in order to set up a canvas and
a GPU handle to draw to. In Rust, this is done through requesting an
HTML div element from the browser. Afterwards, a new canvas element for
WebGPU is pushed to the DOM, as demonstrated in listing
\ref{code:rust-canvas}.

\begin{samepage}
  \lstinputlisting[ float=tp, xleftmargin=0cm,
    % no extra margins for floats
    xrightmargin=0cm,             % no extra margins for floats
    basicstyle=\footnotesize\ttfamily, frame=shadowbox, numbers=left,
    caption={[Code Snippet: Canvas Setup] { Example of setting up a
    canvas element inside the HTML DOM via Rust.
      \imgcredit{Created by the authors.}
    }}, language=Rust, firstnumber=165, label=code:rust-canvas ]
    {listings/wasm/lib_canvas.rs}
\end{samepage}


% --------------------------------------------------------------------------- %
% Section: GPU Handle Acquisition

\section{Step 3 --- GPU Handle Acquisition}
The following stages start with initializing the WebGPU backend,
followed by surface and adapter creation and acquisition of the GPU
handle, as demonstrated in listing \ref{code:rust-handle}.

\begin{samepage}
  \lstinputlisting[ float=tp, xleftmargin=0cm,
    % no extra margins for floats
    xrightmargin=0cm,             % no extra margins for floats
    basicstyle=\footnotesize\ttfamily, frame=shadowbox, numbers=left,
    caption={[Code Snippet: Backend, Surface, Adapter, and GPU Handle
    Setup] { Example code of setting up the initial backend for WebGPU,
    followed by surface creation on the specified window. Afterwards, an
    adapter is created and the handle to the GPU is requested.
      \imgcredit{Created by the authors.}
    }}, language=Rust, firstnumber=23, label=code:rust-handle ]
    {listings/wasm/lib_handle.rs}
\end{samepage}

% --------------------------------------------------------------------------- %
% Section: Shader and Viewport Configuration %

\section{Step 4 --- Shader Generation, Viewport Configuration}
Shader generation is done by including a \emph{*.wgsl} file according to
the WebGPU specification. After setting up the surface configuration
with the appropriate viewport size and texture format, the pipeline can
be created. The shader code can be examined in listing
\ref{code:rust-wgsl}. The required Rust code to include the shader is
shown in listing \ref{code:rust-shader}.


\begin{samepage}
  \lstinputlisting[ float=tp, xleftmargin=0cm,
    % no extra margins for floats
    xrightmargin=0cm,             % no extra margins for floats
    basicstyle=\footnotesize\ttfamily, frame=shadowbox, numbers=left,
    caption={[Code Snippet: WGSL Shader Code] { Example code of a simple
    triangular vertex and fragment shader pair.
      \imgcredit{Created by the authors.}
    }}, language=WGSL, firstnumber=1, label=code:rust-wgsl ]
    {listings/wasm/shader.wgsl}
\end{samepage}

\begin{samepage}
  \lstinputlisting[ float=tp, xleftmargin=0cm,
    % no extra margins for floats
    xrightmargin=0cm,             % no extra margins for floats
    basicstyle=\footnotesize\ttfamily, frame=shadowbox, numbers=left,
    caption={[Code Snippet: Shader and Viewport Setup] { Example code of
    accessing the WebGPU shader inside our Rust example and configuring
    the viewport.
      \imgcredit{Created by the authors.}
    }}, language=Rust, firstnumber=53, label=code:rust-shader ]
    {listings/wasm/lib_shader.rs}
\end{samepage}


% --------------------------------------------------------------------------- %
% Section: Render Function

\section{Step 5 --- Render Function}
Finally, the simple example is completed by setting up a render function
that is called for every frame, as demonstrated in listing
\ref{code:rust-render}.

\begin{samepage}
  \lstinputlisting[ float=tp, xleftmargin=0cm,
    % no extra margins for floats
    xrightmargin=0cm,             % no extra margins for floats
    basicstyle=\footnotesize\ttfamily, frame=shadowbox, numbers=left,
    caption={[Code Snippet: Event Loop Setup] { Example code of setting
    up an event loop to redraw on every frame in our Rust example.
      \imgcredit{Created by the authors.}
    }}, language=Rust, firstnumber=181, label=code:rust-render ]
    {listings/wasm/lib_render.rs}
\end{samepage}
