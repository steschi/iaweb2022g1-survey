%----------------------------------------------------------------
%
%  File    :  fundamentals.tex
%
%  Authors :  Pinheiro de Souza, Schintler, Steinkellner
% 
%  Created :  22 Nov 2022
% 
%----------------------------------------------------------------

\chapter{Fundamentals of Web Graphics}

\label{chap:Fundamentals}

Before diving into WebGPU example code, some fundamental concepts must be introduced.
This section covers the essential stages needed for a minimal example.

\section{Pipelines}
The WebGPU specification lays out GPU commands sequentially, with similarities to WebGL.
It supports two different pipelines:
The \textbf{compute pipeline} is responsible for executing parallel
computations on the GPU hardware. It only consists of a single programmable
stage, customized via the compute shader.
While providing numerous benefits to parallel computations over CPU-executed code, for the sake of brevity, it won't be covered within this survey.
This document mainly focuses on the \textbf{render pipeline}, which houses both the \textbf{vertex shader} and the \textbf{fragment shader}.

\section{Shaders}
Shaders are programmable stages within GPU pipelines, written in a special shader language in order to adhere
to the requirements of the GPU.
The shading language for WebGPU is called WSGL, and will be introduced in chapter \ref{chap:PracticalExample}.
There are two programmable stages within the render pipeline:
The vertex shader and the fragment shader.

\subsection{Vertex Shader}
Vertex shaders are responsible for defining all vertices of a desired primitive.
Each vertex has a position and a set of attributes associated with it.
Vertex coordinates are relative to the centre of the primitive, and must be defined per primitive.
Additional colour values, required by the fragment shader in a later stage of the pipeline, must be appended here to allow pass-through.
Image \ref{fig:vertex-01} illustrates simple example of a vertex shader for a triangular primitive.

\begin{figure}[tp]
\centering
\includegraphics[keepaspectratio,width=\linewidth,height=\halfh]
{images/vertex-01.pdf}

\caption[Vertex shader, example illustration]
{
  Exemplary illustration of a vertex shader for a primitive triangle.
  The vertices V1, V2, V3 define the vertices of the primitive.
\imgcredit{Created by the authors.}
}
\label{fig:vertex-01}
\end{figure}

\subsection{Intermediate Step}
Following the vertex shader, the render pipeline contains an intermediate step that cannot be modified directly.
First, the vertices are assembled into primitives.
These primitives are subsequently clipped to fit within the bounding box of the viewport, before entering the rasterization stage.
This rasterization stage transforms the primitive into rasterized points on the screen.

\subsection{Rasterization}
The rasterization stage is not directly programmable like the vertex or fragment shaders,
but it is a vital part of the rendering pipeline. During this stage, the vertex information
is transformed into rasterized segments on the screen, corresponding to the available pixels in the
viewport. Additionally, this stage includes the culling of obstructed polygons. Any front-facing
polygons are rendered and evaluated, whilst all obstructed ones are discarded.

The most performant way of rasterization is evaluating each available pixel at its centre.
If that centre point is inside a primitive, the corresponding pixel should be coloured.
Whilst being performant and fairly simple to execute, this leads to a stair-like effect on edges that
should be smooth, often called "jaggies".
Figure \ref{fig:rasterization} shows the naive implementation of the rasterization process.
The rasterized points are shaded blue, all discarded pixels are white.

\begin{figure}[tp]
\centering
\includegraphics[keepaspectratio,width=\linewidth,height=\halfh]
{images/rasterization.pdf}

\caption[Rasterization, example illustration]
{
  Exemplary illustration of the rasterization step for a primitive triangle.
  Rasterized points inside the primitive are coloured in blue, whilst points outside are displayed white.
\imgcredit{Created by the authors.}
}
\label{fig:rasterization}
\end{figure}

For most applications, it is desirable to reduce the stair-like jaggies that the rasterization process produces.
This is often done through a process called Anti-Aliasing.
WebGPU provides a built-in way to deal with aliasing by enabling a technique called multisampling for the
rasterization process. This is done through optional configuration during the setup of the scene.

With multisampling enabled, each pixel is evaluated on multiple points, instead of just the centre.
The sample points are deliberately placed near the edge, and are grouped in a sample mask.
The final pixel value is the result of interpolating between all four point samples of the sample mask.

Figure \ref{fig:multisampling} illustrates the process of multisampling on an example triangle.

\begin{figure}[tp]
\centering
\includegraphics[keepaspectratio,width=\linewidth,height=\halfh]
{images/multisampling.pdf}

\caption[Multisampling, example illustration]
{
  Exemplary illustration of multisampling during the rasterization step for a primitive triangle.
  Sample points are indicated with red circles for two of the pixels. All pixels are evaluated with the same
  sample mask, and the final pixel colour value is the result of interpolating between all sample points.
\imgcredit{Created by the authors.}
}
\label{fig:multisampling}
\end{figure}


\subsection{Fragment Shader}
The vertex shader provides vertices that define the shape of a primitive.
Within the fragment shader, this primitive is coloured.
The vertices enter this stage with a defined colour value, and the fragment shader interpolates between these values.
It produces one fragment per rasterization point, and its execution is parallel.
Figure \ref{fig:fragment-01} shows the desired output of this stage.

\begin{figure}[tp]
\centering
\includegraphics[keepaspectratio,width=\linewidth,height=\halfh]
{images/fragment-01.pdf}

\caption[Fragment shader, example illustration]
{
  Exemplary illustration of a fragment shader for a primitive triangle to showcase the desired output.
  Each vertex is defined by its position and an associated colour value.
  For a uniformly coloured triangle, all three colour values must be equal.
\imgcredit{Created by the authors.}
}
\label{fig:fragment-01}
\end{figure}
