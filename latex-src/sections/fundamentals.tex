%----------------------------------------------------------------
%
%  File    :  fundamentals.tex
%
%  Authors :  Pinheiro de Souza, Schintler, Steinkellner
% 
%  Created :  22 Nov 2022
% 
%----------------------------------------------------------------

\chapter{Fundamentals of Web Graphics}

\label{chap:Fundamentals}

Before diving into WebGPU example code, some fundamental concepts must be introduced.
This section covers the essential stages needed for a minimal example.

\section{Pipelines}
The WebGPU specification lays out GPU commands sequentially, with similarities to WebGL.
It supports two different pipelines:
The \textbf{compute pipeline} is responsible for executing parallel
computations on the GPU hardware. It only consists of a single programmable
stage, customized via the compute shader.
While providing numerous benefits to parallel computations over CPU-executed code, for the sake of brevity, it won't be covered within this survey.
This document mainly focuses on the \textbf{render pipeline}, which houses both the \textbf{vertex shader} and the \textbf{fragment shader}.

\section{Shaders}
Shaders are programmable stages within GPU pipelines.
There are two programmable stages within the render pipeline:
The vertex shader and the fragment shader.

\subsection{Vertex Shader}
Vertex shaders are responsible for defining all vertices of a desired primitive.
Each vertex has a position and a set of attributes associated with it.
Vertex coordinates are relative to the centre of the primitive, and must be defined per primitive.
Additional colour values, required by the fragment shader in a later stage of the pipeline, must be appended here to allow pass-through.
Image \ref{fig:vertex-01} illustrates simple example of a vertex shader for a triangular primitive.

\begin{figure}[tp]
\centering
\includegraphics[keepaspectratio,width=\linewidth,height=\halfh]
{images/vertex-01.pdf}

\caption[Vertex shader, example illustration]
{
  Exemplary illustration of a vertex shader for a primitive triangle.
  The vertices V1, V2, V3 define the vertices of the primitive.
\imgcredit{Created by the authors.}
}
\label{fig:vertex-01}
\end{figure}

\subsection{Intermediate Step}
Following the vertex shader, the render pipeline contains an intermediate step that cannot be modified directly.
First, the vertices are assembled into primitives.
These primitives are subsequently clipped to fit within the bounding box of the viewport, before entering the rasterization stage.
This rasterization stage transforms the primitive into rasterized points on the screen.

\subsection{Fragment Shader}
The vertex shader provides vertices that define the shape of a primitive.
Within the fragment shader, this primitive is coloured.
The vertices enter this stage with a defined colour value, and the fragment shader interpolates between these values.
It produces one fragment per rasterization point, and its execution is parallel.
Figure \ref{fig:fragment-01} shows the desired output of this stage, whilst also highlighting the biggest challenge: Rasterization.

\begin{figure}[tp]
\centering
\includegraphics[keepaspectratio,width=\linewidth,height=\halfh]
{images/fragment-01.pdf}

\caption[Fragment shader, example illustration]
{
  Exemplary illustration of a fragment shader for a primitive triangle to showcase the desired output.
  Each vertex is defined by its position and an associated colour value.
  For a uniformly coloured triangle, all three colour values must be equal.
\imgcredit{Created by the authors.}
}
\label{fig:fragment-01}
\end{figure}

