%----------------------------------------------------------------
%
%  File    :  performance.tex
%
%  Authors :  Pinheiro de Souza, Schintler, Steinkellner
% 
%  Created :  22 Nov 2022
% 
%----------------------------------------------------------------

\chapter{Performance}
\label{chap:Performance}

Performance is a crucial aspect of modern web development. Currently, WebGL is used for fast web graphics. WebGPU aims to improve performance in multiple ways. 
The following sections explore and compare performance differences between WebGL and WebGPU. 

\section{Babylon.js}

Since Babylon has already support for both renderers it also provides a benchmark for both.
By rendering the same scene once with WebGL \parencite{babylon_example_webgl} and once with WebGPU \parencite{babylon_example_webgpu}.
Providing also a side-by-side comparison via a video.
There it is visible that WebGPU gives a performance boost in comparison with WebGL.
Showcased by the lower CPU load as well as the higher frames per second (FPS).

\section{Water Simulation}

Another graphics comparison was published by Eytan Manor on the medium webpage \parencite{water_sim_perf}.
Comparing OpenGL and Vulkan the underlying APIs of WebGL and WebGPU. 
In this benchmark, a water mesh gets rendered and transformed to simulate realistic movements.
The results of this benchmark are visible in \ref{fig:water_sim_gpu}.

\begin{figure}[tp]
  \centering
  \includegraphics[keepaspectratio,width=\linewidth,height=\halfh]
  {images/water_sim_fps.png}
  
  \caption[Water Simulation Results --- Average FPS]
  {
  Vulkan to OpenGL benchmark result for Average FPS. 
  \imgcredit{Image taken from: https://eytanmanor.medium.com/the-story-of-webgpu-the-successor-to-webgl-bf5f74bc036a and used under § 42f.(1) of Austrian copyright law}
  }
  \label{fig:water_sim_fps}
\end{figure}

\begin{figure}[tp]
  \centering
  \includegraphics[keepaspectratio,width=\linewidth,height=\halfh]
  {images/water_sim_cpu.png}
  
  \caption[Water Simulation Results --- Average CPU Load]
  {
  Vulkan to OpenGL benchmark result for Average CPU load in \%. 
  \imgcredit{Image taken from: https://eytanmanor.medium.com/the-story-of-webgpu-the-successor-to-webgl-bf5f74bc036a and used under § 42f.(1) of Austrian copyright law}
  }
  \label{fig:water_sim_cpu}
\end{figure}

\begin{figure}[tp]
  \centering
  \includegraphics[keepaspectratio,width=\linewidth,height=\halfh]
  {images/water_sim_gpu.png}
  
  \caption[Water Simulation Results --- Average GPU Load]
  {
  Vulkan to OpenGL benchmark result for Average GPU load in \%. 
  \imgcredit{Image taken from: https://eytanmanor.medium.com/the-story-of-webgpu-the-successor-to-webgl-bf5f74bc036a and used under § 42f.(1) of Austrian copyright law}
  }
  \label{fig:water_sim_gpu}
\end{figure}

These results show it quite well that Vulkan and thus WebGPU give us a great performance boost at a much lower CPU cost.
In contrast to OpenGL, Vulkan offers a low-level API that allows developers to parallelize workloads.
It provides higher performance, visible in the FPS, and more balanced CPU to GPU usage.

\section{Matrix Multiplication}

One benchmark that also should not be omitted is a comparison of computation tasks
since WebGPU is equipped with a compute pipeline and not only with a render pipeline.\\
In WebGL, there is not something similar to WebGPUs compute pipeline.
However, it is possible to mimic the compute pipeline with the help of a little trick.
The data used for computations need to be converted into an image.
This image is then passed to the pipeline and the fragment shader, where the actual computations are performed.
After the computations are done the results are stored in the same image as pixel color values.
Those values now have to be read from the image back into the original data format. \\
In contrast, WebGPU provides compute shaders for those tasks.
The data is passed to the GPU with a buffer and the computations are done asynchronously. 
Additionally, there is no need to convert or extract the data from an image, giving another performance boost.

Such a benchmark is visible in \ref{fig:matrix_mult}. 
In this benchmark, the size of the matrix gets continuously bigger. It demonstrates that the "Image hack" with WebGL is approximately 3.5 times slower than the compute shaders of WebGPU.
Additionally, it can process more data and it does not block the main thread.


\begin{figure}[tp]
  \centering
  \includegraphics[keepaspectratio,width=\linewidth,height=\halfh]
  {images/matrix_mult.png}
  
  \caption[Matrix Multiplication Benchmark]
  {
  \imgcredit{Image taken from: https://pixelscommander.com/javascript/webgpu-computations-performance-in-comparison-to-webgl/ and used under § 42f.(1) of Austrian copyright law}
  }
  \label{fig:matrix_mult}
\end{figure}


