%----------------------------------------------------------------
%
%  File    :  performance.tex
%
%  Authors :  Pinheiro de Souza, Schintler, Steinkellner
% 
%  Created :  22 Nov 2022
% 
%----------------------------------------------------------------

\chapter{Performance}

If WebGPU wants to become the need standard and supersede WebGL it has to provide the performance.

\section{Babylon.js}

Since Babylon has already support for both renderers it also provides a benchmark for both.
By rendering the same scene once with WebGL \parencite{babylon_example_webgl} and once with WebGPU \parencite{babylon_example_webgpu}.
Providing also a side by side comparison via a video.
There it is visible that WebGPU gives a performance boost in comparison with WebGL.
Showcased by the lower CPU load as well as the higher frames per seconds (FPS).

\section{Water simulation}

Another graphics comparison was published by Eytan Manor on the medium webpage \parencite{water_sim_perf}.
Comparing OpgenGL and Vulkan the underlying APIs of WebGL and WebGPU. 
In this benchmark, a water mesh gets renderderd and transformed to simulate realistic movements.
The results of this benchmark are visible in \ref{fig:water_sim_gpu}.

\begin{figure}[tp]
  \centering
  \includegraphics[keepaspectratio,width=\linewidth,height=\halfh]
  {images/water_sim_fps.png}
  
  \caption[Water simulation results average frames per second (FPS)]
  {
  \imgcredit{https://eytanmanor.medium.com/the-story-of-webgpu-the-successor-to-webgl-bf5f74bc036a}
  }
  \label{fig:water_sim_fps}
\end{figure}

\begin{figure}[tp]
  \centering
  \includegraphics[keepaspectratio,width=\linewidth,height=\halfh]
  {images/water_sim_cpu.png}
  
  \caption[Water simulation results average frames per second (FPS)]
  {
  \imgcredit{https://eytanmanor.medium.com/the-story-of-webgpu-the-successor-to-webgl-bf5f74bc036a}
  }
  \label{fig:water_sim_cpu}
\end{figure}

\begin{figure}[tp]
  \centering
  \includegraphics[keepaspectratio,width=\linewidth,height=\halfh]
  {images/water_sim_gpu.png}
  
  \caption[Water simulation results average frames per second (FPS)]
  {
  \imgcredit{https://eytanmanor.medium.com/the-story-of-webgpu-the-successor-to-webgl-bf5f74bc036a}
  }
  \label{fig:water_sim_gpu}
\end{figure}

This results show it quite well that Vulkan and thus WebGPU gives us a great performance boost at much less CPU cost.
Since Vulkan is considerable a lower-level API than OpenGL and also providing developers better options to distribute work through multiple CPU cores.
It provides higher performance, visible in the FPS, and more balanced CPU to GPU usage.

\section{Matrix multiplication}

One benchmark that also should not be omitted is a comparison of computations task.
Since WebGPU is equipped with a compute pipeline and not only with a render pipeline.\\
In WebGL there is not something similiar to WebGPUs compute pipeline.
However it is possible to mimic the compute pipeline with the help of a little trick.
The data where computations should be performed on needs to be converted into an image.
This image is than passed to the pipeline and the fragment shader, where the actual computations are performed.
After the computations are down the results get stored in the same image as pixel color values.
Those values now have to be read from the image back into the original dataformat from the image. \\
In contrast, WebGPU provides compute shaders for those tasks.
The data is passed to the GPU with a buffer and the computations are done asynchronously. 
Additionally there is no need to convert or extract the data from an image, giving another performance boost.

Such a benchmark is visible in \ref{fig:matrix_mult}. 
In this benchmark the size of the matrix gets continuously bigger.

\begin{figure}[tp]
  \centering
  \includegraphics[keepaspectratio,width=\linewidth,height=\halfh]
  {images/matrix_mult.png}
  
  \caption[Water simulation results average frames per second (FPS)]
  {
  \imgcredit{https://eytanmanor.medium.com/the-story-of-webgpu-the-successor-to-webgl-bf5f74bc036a}
  }
  \label{fig:matrix_mult}
\end{figure}

In this example it is visible to see that the "Image hack" with WebGL is approximatly 3.5 times slower than the compute shaders of WebGPU.
Additionally it can process more data and it does not block the main thread.
Showcasing promising results for WebGPU and its compute shaders.


\label{chap:Performance}

\begin{displayquote}
  This is a very elaborate quote by me.
\end{displayquote}
