%----------------------------------------------------------------
%
%  File    :  libraries.tex
%
%  Authors :  Pinheiro de Souza, Schintler, Steinkellner
% 
%  Created :  22 Nov 2022
% 
%----------------------------------------------------------------

\newcommand*{\codeThreejs}[1]{
  \inputminted[
    frame=lines,
    framesep=2mm,
    baselinestretch=1.2,
    linenos,
    fontsize=\scriptsize,
    breaklines=true,
    highlightlines={30-39,43-49,81-84},
    firstline=30,
    lastline=84
  ]{typescript}{#1}}

\newcommand*{\codeBabylon}[1]{
  \inputminted[
    frame=lines,
    framesep=2mm,
    baselinestretch=1.2,
    linenos,
    fontsize=\scriptsize,
    breaklines=true,
    highlightlines={21-27,39-45},
    firstline=21,
    lastline=45
  ]{typescript}{#1}}

\chapter{Libraries}

\label{chap:Libraries}

Getting started with WebGL is quite simple. 
One factor that makes it that simple are the many libraries that exist and provide an interface for WebGL.
However the support of most libraries for WebGPU is still lacking even though we should expect a release this quarter.

\section{Two.js}

Two.js is one of the most prominent library for working in 2D, hence the name.
Providing a drawing api for drawing in multiple contexts, e.g. svg, canvas and WebGL.\\
On the forum of Two.js the developers provide a little inside into the development.
Talking about the next steps of the library and if WebGPU will be supported.
Stating that \begin{displayquote} "... a WebGPU renderer seems like the natural evolution of renderers in Two.js." \end{displayquote}\parencite{two_quote}.
However also pointing to the fact that WebGPU is not released yet and thus not starting to work on it until there is an official release.

\section{Three.js}

Three.js is as the name suggests a 3D JavaScript library. 
What Two.js is in the 2D domain in the 3D domain there is Three.js.\\
In contrast to Two.js, Three.js is already actively working on implementing a WebGPU renderer.
Currently there is already a WebGPU renderer implemented in Three.js that is usable. 
Even though there are still some issues and not all of the features work as intended, it is still usable to a certain degree.
Which is also visible by the several different examples that are already online \parencite{three_examples}.
As well as in our own example that we created using Three.js and the WebGPU renderer. \\

Listing \ref{code:three-scene} shows the required setup for a simple scene in Three.js. \\
First, one has to create each object invidually. Afterwards, they can be added to the scene.

\begin{listing}
  \centering
  \codeThreejs{../software/packages/threejs-example/src/main.ts}
  \caption[Code Snippet: Three.js Example]
  {
    Scene, camera and WebGPU setup in Three.js, followed by simple object creation.
    \imgcredit{Created by the authors.}
  }
  \label{code:three-scene}
\end{listing}


After adding hover interactivity it is possible to get create a scene that reacts to the position of the mouse cursor.
The initial state of the scene can be seen in figure \ref{fig:three_img1}. The results of the hover effect are visible in figure \ref{fig:three_img2}.

\begin{figure}[tp]
  \centering
  \includegraphics[keepaspectratio,width=\linewidth,height=\halfh]
  {images/three_example_img1.png}
  
  \caption[Three.js Example Without Mouse Interactivity]
  {
  A simple Three.js example without the mouse hovering over any object.
  \imgcredit{Created by the authors.}
  }
  \label{fig:three_img1}
\end{figure}

\begin{figure}[tp]
  \centering
  \includegraphics[keepaspectratio,width=\linewidth,height=\halfh]
  {images/three_example_img2.png}
  
  \caption[Three.js Example With Mouse Interactivity]
  {
  A simple Three.js example with the mouse hovering over multiple objects.
  \imgcredit{Created by the authors.}
  }
  \label{fig:three_img2}
\end{figure}

\section{Babylon.js}

Last but not least there is Babylon.js.
Babylon.js started the integration process for WebGPU already back in 2019 \parencite{babylon_start_webgpu}. 
The first usable WebGPU renderer in Babylon.js was released with version 5.0 of the library \parencite{babylon_released}.
Another milestone that Babylon.js was able to achieve was to got feature parity with the WebGL renderer back in January 2022 \parencite{babylon_parity}.
Thus providing full WebGPU support for the library.

Similar to Three.js it is possible to create an example scene quite easily.
First, one has to create a scene and specify the engine one wants to use.
After this we can start creating and adding objects to the scene, as shown in listing \ref{code:babylon-scene}.

\begin{listing}
  \centering
  \codesnippet{19}{55}{typescript}{../software/packages/babylon-js-example/src/main.ts}
  \caption[Code Snippet: Babylon Example]
  {
    Scene, camera and WebGPU setup in Babylon.js, followed by simple object creation.
    \imgcredit{Created by the authors.}
  }
  \label{code:babylon-scene}
\end{listing}

Figure \ref{fig:babylon_example_img1} shows an interactive example of a Babylon.js scene where the user can drag and drop
a cube.
\begin{figure}[tp]
  \centering
  \includegraphics[keepaspectratio,width=\linewidth,height=\halfh]
  {images/babylon_example_img1.png}
  
  \caption[Babylon.js Example With Mouse Interactivity]
  {
  Simple interactive example of a Babylon.js scene where the user can drag and drop a cube.
  \imgcredit{Created by the authors.}
  }
  \label{fig:babylon_example_img1}
\end{figure}

