%----------------------------------------------------------------
%
%  File    :  practical-example.tex
%
%  Authors :  Pinheiro de Souza, Schintler, Steinkellner
% 
%  Created :  22 Nov 2022
% 
%----------------------------------------------------------------

\newcommand*{\codesnippet}[3]{
  \inputminted[
    frame=lines,
    framesep=2mm,
    baselinestretch=1.2,
    linenos,
    fontsize=\scriptsize,
    firstline=#1,
    lastline=#2
  ]{typescript}{#3}}


\newcommand*{\code}[1]{
  \inputminted[
    frame=lines,
    framesep=2mm,
    baselinestretch=1.2,
    linenos,
    fontsize=\scriptsize,
    breaklines=true,
    highlightlines={30-33,42-49}
  ]{typescript}{#1}}


\chapter{Practical Example of using WebGPU}

\label{chap:PracticalExample}

Due to the nature of rendering directly with a GPU using WebGPU is a rather tedious process.
This section covers the essential stages needed for a minimal working example.

\section{General Flow}

The inherent complexity of programming fast Web Graphics via WebGPU stems from the way one has to interact with a GPU itself.
In contrast to simpler 2D graphics like SVG and Canvas2D WebGPU is just an abstraction layer on top of broadly used GPU APIs like Vulkan (cite?).
It, therefore, has to perform similar steps as native software rendering 3D scenes. As can be seen in \ref*{fig:webgpu-explain} we first need to
collect all information of a scene, like Vertex positions and color, as well as the matching WGSL Shader Code and mangle it via the CPU into a properly structured
buffer on the GPU's V-RAM. Only then can the CPU hand over control and allow the GPU to start execution of the WGSL Shader Code which will use the buffer to render a proper scene.
This workflow is described in more detail in the following sections.


\begin{figure}[tp]
  \centering
  \includegraphics[keepaspectratio,width=\linewidth,height=\halfh]
  {images/wgpu-explain.pdf}

  \caption[Dataflow in WebGPU, example]
  {
    An exemplary illustration of how instructions and data of a 3d scene have to be handled
    to use WebGPU.
    \imgcredit{Created by the authors.}
  }
  \label{fig:webgpu-explain}
\end{figure}


\section{Steps}




\subsection{Step 1 - Creating a Buffer}

As can be seen in \ref*{code:create-buffer} ... TODO


\begin{figure}[tp]
  \centering
  \codesnippet{4}{22}{../webgpu-simple/src/helpers.ts}

  \caption[Code Snippet: createBuffer]
  {
    An exemplary illustration of how instructions and data of a 3d scene have to be handled
    to use WebGPU.
    \imgcredit{Created by the authors.}
  }
  \label{code:create-buffer}
\end{figure}


\subsection{Step 2 - Encoding the Vertex Information}
As can be seen in \ref*{code:vertex-encode} ... TODO


\begin{figure}[tp]

  \centering
  \code{../shared/src/vertex.ts}

  \caption[Code Snippet: Vertex Encoding]
  {
    An exemplary illustration of how to encode vertex information for use in WebGPU
    \imgcredit{Created by the authors.}
  }
  \label{code:vertex-encode}
\end{figure}

\subsection{Step 3 - Setting up the Pipeline}

As can be seen in \ref*{code:pipeline-setup} ... TODO


\begin{figure}[tp]

  \centering
  \codesnippet{31}{49}{../webgpu-simple/src/main.ts}

  \caption[Code Snippet: WebGPU Pipeline]
  {
    An exemplary illustration of how to setup a WebGPU pipeline
    \imgcredit{Created by the authors.}
  }
  \label{code:pipeline-setup}
\end{figure}

\subsection{Step 4 - Adding the WGSL Shader Code}


