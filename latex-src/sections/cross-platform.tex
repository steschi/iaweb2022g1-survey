%----------------------------------------------------------------
%
%  File    :  cross-platform.tex
%
%  Authors :  Pinheiro de Souza, Schintler, Steinkellner
% 
%  Created :  22 Nov 2022
% 
%----------------------------------------------------------------

\chapter{Cross-Platform}

\label{chap:Cross-Platform}

\section{WebAssembly}
Instead of writing strictly platform-dependent native code, WebAssembly can be used to target
execution within the context of the browser \parencite{wasm}. It offers a load-time-efficient virtual stack
machine, executed within the sandbox of a browser tab. 
It serves as a compile target for numerous languages, such as C, C++, or Rust \parencite{ivis-2022}.
The compilation ends with a single executable file with the \emph{*.wasm} format,
which can be loaded and executed within the browser.

Most modern browsers already support the use of WebAssembly, with a 96\% coverage across
global internet users, as shown in figure \ref{fig:wasm}.

\begin{figure}[tp]
\centering
\includegraphics[keepaspectratio,width=\linewidth,height=\halfh]
{images/wasm.png}

\caption[Browser Support of WebAssembly]{
  WebAssembly is supported by all major browsers used by about 96\% of global internet users.
\imgcredit{Screenshot taken by the authors of this paper.}
}
\label{fig:wasm}
\end{figure}

\section{Engines}

Using WebAssembly in conjunction with WebGPU allows the use of powerful and reliable Engines usually used for native development. 
A few of the most recognized engines already have expressed interest in supporting WebGPU. 


\subsection{Godot}
Godot is a free and open-source game engine under the MIT license \parencite{godot}. Based on communication on the project's issue board contributors plan to drop WebGL support in favor of WebGPU support \parencite{godot_webpu_support}.

\subsection{PlayCanvas}
PlayCanvas is a proprietary game engine specifically designed for the web \parencite{playcanvas}. It used to be focused on WebGL but is currently in the process of transitioning its codebase to support WebGPU.

\subsection{Unity}
Unity is a paid game engine and is especially used on mobile devices \parencite{unity}. Based on the job listings of the parent company one can assume they are actively looking for people to work on supporting WebGPU. 

\subsection{Unreal Engine}

Unreal Engine is an open-source but paid game engine used in games, movies and simulations \parencite{unreal}. It did support WebGL but plans to drop support for any web-based platform as it wants to focus on native development only. 